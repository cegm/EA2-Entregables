\documentclass[10pt, spanish]{article}

    \usepackage[breakable]{tcolorbox}
    \usepackage{parskip} % Stop auto-indenting (to mimic markdown behaviour)

    \usepackage{iftex}
    \ifPDFTeX
    	\usepackage[T1]{fontenc}
    	\usepackage{mathpazo}
    \else
    	\usepackage{fontspec}
    \fi

    % Basic figure setup, for now with no caption control since it's done
    % automatically by Pandoc (which extracts ![](path) syntax from Markdown).
    \usepackage{graphicx}
    % Maintain compatibility with old templates. Remove in nbconvert 6.0
    \let\Oldincludegraphics\includegraphics
    % Ensure that by default, figures have no caption (until we provide a
    % proper Figure object with a Caption API and a way to capture that
    % in the conversion process - todo).
    \usepackage{caption}
    \DeclareCaptionFormat{nocaption}{}
    \captionsetup{format=nocaption,aboveskip=0pt,belowskip=0pt}

    \usepackage[Export]{adjustbox} % Used to constrain images to a maximum size
    \adjustboxset{max size={0.9\linewidth}{0.9\paperheight}}
    \usepackage{float}
    \floatplacement{figure}{H} % forces figures to be placed at the correct location
    %\usepackage{xcolor} % Allow colors to be defined

    \usepackage{enumerate} % Needed for markdown enumerations to work
    \usepackage{geometry} % Used to adjust the document margins
    \usepackage{amsmath} % Equations
    \usepackage{amssymb} % Equations
    \usepackage{textcomp} % defines textquotesingle
    % Hack from http://tex.stackexchange.com/a/47451/13684:
    \AtBeginDocument{%
        \def\PYZsq{\textquotesingle}% Upright quotes in Pygmentized code
    }
    \usepackage{upquote} % Upright quotes for verbatim code
    \usepackage{eurosym} % defines \euro
    \usepackage[mathletters]{ucs} % Extended unicode (utf-8) support
    \usepackage{fancyvrb} % verbatim replacement that allows latex
    \usepackage{grffile} % extends the file name processing of package graphics
                         % to support a larger range
    \makeatletter % fix for grffile with XeLaTeX
    \def\Gread@@xetex#1{%
      \IfFileExists{"\Gin@base".bb}%
      {\Gread@eps{\Gin@base.bb}}%
      {\Gread@@xetex@aux#1}%
    }
    \makeatother

    % The hyperref package gives us a pdf with properly built
    % internal navigation ('pdf bookmarks' for the table of contents,
    % internal cross-reference links, web links for URLs, etc.)
    \usepackage{hyperref}
    % The default LaTeX title has an obnoxious amount of whitespace. By default,
    % titling removes some of it. It also provides customization options.
    \usepackage{titling}
    \usepackage{longtable} % longtable support required by pandoc >1.10
    \usepackage{booktabs}  % table support for pandoc > 1.12.2
    \usepackage[inline]{enumitem} % IRkernel/repr support (it uses the enumerate* environment)
    \usepackage[normalem]{ulem} % ulem is needed to support strikethroughs (\sout)
                                % normalem makes italics be italics, not underlines
    \usepackage{mathrsfs}



    % Colors for the hyperref package
    \definecolor{urlcolor}{rgb}{0,.145,.698}
    \definecolor{linkcolor}{rgb}{0,0,0}
    \definecolor{citecolor}{rgb}{.12,.54,.11}

    % ANSI colors
    \definecolor{ansi-black}{HTML}{3E424D}
    \definecolor{ansi-black-intense}{HTML}{282C36}
    \definecolor{ansi-red}{HTML}{E75C58}
    \definecolor{ansi-red-intense}{HTML}{B22B31}
    \definecolor{ansi-green}{HTML}{00A250}
    \definecolor{ansi-green-intense}{HTML}{007427}
    \definecolor{ansi-yellow}{HTML}{DDB62B}
    \definecolor{ansi-yellow-intense}{HTML}{B27D12}
    \definecolor{ansi-blue}{HTML}{208FFB}
    \definecolor{ansi-blue-intense}{HTML}{0065CA}
    \definecolor{ansi-magenta}{HTML}{D160C4}
    \definecolor{ansi-magenta-intense}{HTML}{A03196}
    \definecolor{ansi-cyan}{HTML}{60C6C8}
    \definecolor{ansi-cyan-intense}{HTML}{258F8F}
    \definecolor{ansi-white}{HTML}{C5C1B4}
    \definecolor{ansi-white-intense}{HTML}{A1A6B2}
    \definecolor{ansi-default-inverse-fg}{HTML}{FFFFFF}
    \definecolor{ansi-default-inverse-bg}{HTML}{000000}

    % commands and environments needed by pandoc snippets
    % extracted from the output of `pandoc -s`
    \providecommand{\tightlist}{%
      \setlength{\itemsep}{0pt}\setlength{\parskip}{0pt}}
    \DefineVerbatimEnvironment{Highlighting}{Verbatim}{commandchars=\\\{\}}
    % Add ',fontsize=\small' for more characters per line
    \newenvironment{Shaded}{}{}
    \newcommand{\KeywordTok}[1]{\textcolor[rgb]{0.00,0.44,0.13}{\textbf{{#1}}}}
    \newcommand{\DataTypeTok}[1]{\textcolor[rgb]{0.56,0.13,0.00}{{#1}}}
    \newcommand{\DecValTok}[1]{\textcolor[rgb]{0.25,0.63,0.44}{{#1}}}
    \newcommand{\BaseNTok}[1]{\textcolor[rgb]{0.25,0.63,0.44}{{#1}}}
    \newcommand{\FloatTok}[1]{\textcolor[rgb]{0.25,0.63,0.44}{{#1}}}
    \newcommand{\CharTok}[1]{\textcolor[rgb]{0.25,0.44,0.63}{{#1}}}
    \newcommand{\StringTok}[1]{\textcolor[rgb]{0.25,0.44,0.63}{{#1}}}
    \newcommand{\CommentTok}[1]{\textcolor[rgb]{0.38,0.63,0.69}{\textit{{#1}}}}
    \newcommand{\OtherTok}[1]{\textcolor[rgb]{0.00,0.44,0.13}{{#1}}}
    \newcommand{\AlertTok}[1]{\textcolor[rgb]{1.00,0.00,0.00}{\textbf{{#1}}}}
    \newcommand{\FunctionTok}[1]{\textcolor[rgb]{0.02,0.16,0.49}{{#1}}}
    \newcommand{\RegionMarkerTok}[1]{{#1}}
    \newcommand{\ErrorTok}[1]{\textcolor[rgb]{1.00,0.00,0.00}{\textbf{{#1}}}}
    \newcommand{\NormalTok}[1]{{#1}}

    % Additional commands for more recent versions of Pandoc
    \newcommand{\ConstantTok}[1]{\textcolor[rgb]{0.53,0.00,0.00}{{#1}}}
    \newcommand{\SpecialCharTok}[1]{\textcolor[rgb]{0.25,0.44,0.63}{{#1}}}
    \newcommand{\VerbatimStringTok}[1]{\textcolor[rgb]{0.25,0.44,0.63}{{#1}}}
    \newcommand{\SpecialStringTok}[1]{\textcolor[rgb]{0.73,0.40,0.53}{{#1}}}
    \newcommand{\ImportTok}[1]{{#1}}
    \newcommand{\DocumentationTok}[1]{\textcolor[rgb]{0.73,0.13,0.13}{\textit{{#1}}}}
    \newcommand{\AnnotationTok}[1]{\textcolor[rgb]{0.38,0.63,0.69}{\textbf{\textit{{#1}}}}}
    \newcommand{\CommentVarTok}[1]{\textcolor[rgb]{0.38,0.63,0.69}{\textbf{\textit{{#1}}}}}
    \newcommand{\VariableTok}[1]{\textcolor[rgb]{0.10,0.09,0.49}{{#1}}}
    \newcommand{\ControlFlowTok}[1]{\textcolor[rgb]{0.00,0.44,0.13}{\textbf{{#1}}}}
    \newcommand{\OperatorTok}[1]{\textcolor[rgb]{0.40,0.40,0.40}{{#1}}}
    \newcommand{\BuiltInTok}[1]{{#1}}
    \newcommand{\ExtensionTok}[1]{{#1}}
    \newcommand{\PreprocessorTok}[1]{\textcolor[rgb]{0.74,0.48,0.00}{{#1}}}
    \newcommand{\AttributeTok}[1]{\textcolor[rgb]{0.49,0.56,0.16}{{#1}}}
    \newcommand{\InformationTok}[1]{\textcolor[rgb]{0.38,0.63,0.69}{\textbf{\textit{{#1}}}}}
    \newcommand{\WarningTok}[1]{\textcolor[rgb]{0.38,0.63,0.69}{\textbf{\textit{{#1}}}}}


    % Define a nice break command that doesn't care if a line doesn't already
    % exist.
    \def\br{\hspace*{\fill} \\* }
    % Math Jax compatibility definitions
    \def\gt{>}
    \def\lt{<}
    \let\Oldtex\TeX
    \let\Oldlatex\LaTeX
    \renewcommand{\TeX}{\textrm{\Oldtex}}
    \renewcommand{\LaTeX}{\textrm{\Oldlatex}}
    % Document parameters
    % Document title
    \title{Control 3}
    \date{21 de octubre de 2020}
    \author{KND}





% Pygments definitions
\makeatletter
\def\PY@reset{\let\PY@it=\relax \let\PY@bf=\relax%
    \let\PY@ul=\relax \let\PY@tc=\relax%
    \let\PY@bc=\relax \let\PY@ff=\relax}
\def\PY@tok#1{\csname PY@tok@#1\endcsname}
\def\PY@toks#1+{\ifx\relax#1\empty\else%
    \PY@tok{#1}\expandafter\PY@toks\fi}
\def\PY@do#1{\PY@bc{\PY@tc{\PY@ul{%
    \PY@it{\PY@bf{\PY@ff{#1}}}}}}}
\def\PY#1#2{\PY@reset\PY@toks#1+\relax+\PY@do{#2}}

\expandafter\def\csname PY@tok@w\endcsname{\def\PY@tc##1{\textcolor[rgb]{0.73,0.73,0.73}{##1}}}
\expandafter\def\csname PY@tok@c\endcsname{\let\PY@it=\textit\def\PY@tc##1{\textcolor[rgb]{0.25,0.50,0.50}{##1}}}
\expandafter\def\csname PY@tok@cp\endcsname{\def\PY@tc##1{\textcolor[rgb]{0.74,0.48,0.00}{##1}}}
\expandafter\def\csname PY@tok@k\endcsname{\let\PY@bf=\textbf\def\PY@tc##1{\textcolor[rgb]{0.00,0.50,0.00}{##1}}}
\expandafter\def\csname PY@tok@kp\endcsname{\def\PY@tc##1{\textcolor[rgb]{0.00,0.50,0.00}{##1}}}
\expandafter\def\csname PY@tok@kt\endcsname{\def\PY@tc##1{\textcolor[rgb]{0.69,0.00,0.25}{##1}}}
\expandafter\def\csname PY@tok@o\endcsname{\def\PY@tc##1{\textcolor[rgb]{0.40,0.40,0.40}{##1}}}
\expandafter\def\csname PY@tok@ow\endcsname{\let\PY@bf=\textbf\def\PY@tc##1{\textcolor[rgb]{0.67,0.13,1.00}{##1}}}
\expandafter\def\csname PY@tok@nb\endcsname{\def\PY@tc##1{\textcolor[rgb]{0.00,0.50,0.00}{##1}}}
\expandafter\def\csname PY@tok@nf\endcsname{\def\PY@tc##1{\textcolor[rgb]{0.00,0.00,1.00}{##1}}}
\expandafter\def\csname PY@tok@nc\endcsname{\let\PY@bf=\textbf\def\PY@tc##1{\textcolor[rgb]{0.00,0.00,1.00}{##1}}}
\expandafter\def\csname PY@tok@nn\endcsname{\let\PY@bf=\textbf\def\PY@tc##1{\textcolor[rgb]{0.00,0.00,1.00}{##1}}}
\expandafter\def\csname PY@tok@ne\endcsname{\let\PY@bf=\textbf\def\PY@tc##1{\textcolor[rgb]{0.82,0.25,0.23}{##1}}}
\expandafter\def\csname PY@tok@nv\endcsname{\def\PY@tc##1{\textcolor[rgb]{0.10,0.09,0.49}{##1}}}
\expandafter\def\csname PY@tok@no\endcsname{\def\PY@tc##1{\textcolor[rgb]{0.53,0.00,0.00}{##1}}}
\expandafter\def\csname PY@tok@nl\endcsname{\def\PY@tc##1{\textcolor[rgb]{0.63,0.63,0.00}{##1}}}
\expandafter\def\csname PY@tok@ni\endcsname{\let\PY@bf=\textbf\def\PY@tc##1{\textcolor[rgb]{0.60,0.60,0.60}{##1}}}
\expandafter\def\csname PY@tok@na\endcsname{\def\PY@tc##1{\textcolor[rgb]{0.49,0.56,0.16}{##1}}}
\expandafter\def\csname PY@tok@nt\endcsname{\let\PY@bf=\textbf\def\PY@tc##1{\textcolor[rgb]{0.00,0.50,0.00}{##1}}}
\expandafter\def\csname PY@tok@nd\endcsname{\def\PY@tc##1{\textcolor[rgb]{0.67,0.13,1.00}{##1}}}
\expandafter\def\csname PY@tok@s\endcsname{\def\PY@tc##1{\textcolor[rgb]{0.73,0.13,0.13}{##1}}}
\expandafter\def\csname PY@tok@sd\endcsname{\let\PY@it=\textit\def\PY@tc##1{\textcolor[rgb]{0.73,0.13,0.13}{##1}}}
\expandafter\def\csname PY@tok@si\endcsname{\let\PY@bf=\textbf\def\PY@tc##1{\textcolor[rgb]{0.73,0.40,0.53}{##1}}}
\expandafter\def\csname PY@tok@se\endcsname{\let\PY@bf=\textbf\def\PY@tc##1{\textcolor[rgb]{0.73,0.40,0.13}{##1}}}
\expandafter\def\csname PY@tok@sr\endcsname{\def\PY@tc##1{\textcolor[rgb]{0.73,0.40,0.53}{##1}}}
\expandafter\def\csname PY@tok@ss\endcsname{\def\PY@tc##1{\textcolor[rgb]{0.10,0.09,0.49}{##1}}}
\expandafter\def\csname PY@tok@sx\endcsname{\def\PY@tc##1{\textcolor[rgb]{0.00,0.50,0.00}{##1}}}
\expandafter\def\csname PY@tok@m\endcsname{\def\PY@tc##1{\textcolor[rgb]{0.40,0.40,0.40}{##1}}}
\expandafter\def\csname PY@tok@gh\endcsname{\let\PY@bf=\textbf\def\PY@tc##1{\textcolor[rgb]{0.00,0.00,0.50}{##1}}}
\expandafter\def\csname PY@tok@gu\endcsname{\let\PY@bf=\textbf\def\PY@tc##1{\textcolor[rgb]{0.50,0.00,0.50}{##1}}}
\expandafter\def\csname PY@tok@gd\endcsname{\def\PY@tc##1{\textcolor[rgb]{0.63,0.00,0.00}{##1}}}
\expandafter\def\csname PY@tok@gi\endcsname{\def\PY@tc##1{\textcolor[rgb]{0.00,0.63,0.00}{##1}}}
\expandafter\def\csname PY@tok@gr\endcsname{\def\PY@tc##1{\textcolor[rgb]{1.00,0.00,0.00}{##1}}}
\expandafter\def\csname PY@tok@ge\endcsname{\let\PY@it=\textit}
\expandafter\def\csname PY@tok@gs\endcsname{\let\PY@bf=\textbf}
\expandafter\def\csname PY@tok@gp\endcsname{\let\PY@bf=\textbf\def\PY@tc##1{\textcolor[rgb]{0.00,0.00,0.50}{##1}}}
\expandafter\def\csname PY@tok@go\endcsname{\def\PY@tc##1{\textcolor[rgb]{0.53,0.53,0.53}{##1}}}
\expandafter\def\csname PY@tok@gt\endcsname{\def\PY@tc##1{\textcolor[rgb]{0.00,0.27,0.87}{##1}}}
\expandafter\def\csname PY@tok@err\endcsname{\def\PY@bc##1{\setlength{\fboxsep}{0pt}\fcolorbox[rgb]{1.00,0.00,0.00}{1,1,1}{\strut ##1}}}
\expandafter\def\csname PY@tok@kc\endcsname{\let\PY@bf=\textbf\def\PY@tc##1{\textcolor[rgb]{0.00,0.50,0.00}{##1}}}
\expandafter\def\csname PY@tok@kd\endcsname{\let\PY@bf=\textbf\def\PY@tc##1{\textcolor[rgb]{0.00,0.50,0.00}{##1}}}
\expandafter\def\csname PY@tok@kn\endcsname{\let\PY@bf=\textbf\def\PY@tc##1{\textcolor[rgb]{0.00,0.50,0.00}{##1}}}
\expandafter\def\csname PY@tok@kr\endcsname{\let\PY@bf=\textbf\def\PY@tc##1{\textcolor[rgb]{0.00,0.50,0.00}{##1}}}
\expandafter\def\csname PY@tok@bp\endcsname{\def\PY@tc##1{\textcolor[rgb]{0.00,0.50,0.00}{##1}}}
\expandafter\def\csname PY@tok@fm\endcsname{\def\PY@tc##1{\textcolor[rgb]{0.00,0.00,1.00}{##1}}}
\expandafter\def\csname PY@tok@vc\endcsname{\def\PY@tc##1{\textcolor[rgb]{0.10,0.09,0.49}{##1}}}
\expandafter\def\csname PY@tok@vg\endcsname{\def\PY@tc##1{\textcolor[rgb]{0.10,0.09,0.49}{##1}}}
\expandafter\def\csname PY@tok@vi\endcsname{\def\PY@tc##1{\textcolor[rgb]{0.10,0.09,0.49}{##1}}}
\expandafter\def\csname PY@tok@vm\endcsname{\def\PY@tc##1{\textcolor[rgb]{0.10,0.09,0.49}{##1}}}
\expandafter\def\csname PY@tok@sa\endcsname{\def\PY@tc##1{\textcolor[rgb]{0.73,0.13,0.13}{##1}}}
\expandafter\def\csname PY@tok@sb\endcsname{\def\PY@tc##1{\textcolor[rgb]{0.73,0.13,0.13}{##1}}}
\expandafter\def\csname PY@tok@sc\endcsname{\def\PY@tc##1{\textcolor[rgb]{0.73,0.13,0.13}{##1}}}
\expandafter\def\csname PY@tok@dl\endcsname{\def\PY@tc##1{\textcolor[rgb]{0.73,0.13,0.13}{##1}}}
\expandafter\def\csname PY@tok@s2\endcsname{\def\PY@tc##1{\textcolor[rgb]{0.73,0.13,0.13}{##1}}}
\expandafter\def\csname PY@tok@sh\endcsname{\def\PY@tc##1{\textcolor[rgb]{0.73,0.13,0.13}{##1}}}
\expandafter\def\csname PY@tok@s1\endcsname{\def\PY@tc##1{\textcolor[rgb]{0.73,0.13,0.13}{##1}}}
\expandafter\def\csname PY@tok@mb\endcsname{\def\PY@tc##1{\textcolor[rgb]{0.40,0.40,0.40}{##1}}}
\expandafter\def\csname PY@tok@mf\endcsname{\def\PY@tc##1{\textcolor[rgb]{0.40,0.40,0.40}{##1}}}
\expandafter\def\csname PY@tok@mh\endcsname{\def\PY@tc##1{\textcolor[rgb]{0.40,0.40,0.40}{##1}}}
\expandafter\def\csname PY@tok@mi\endcsname{\def\PY@tc##1{\textcolor[rgb]{0.40,0.40,0.40}{##1}}}
\expandafter\def\csname PY@tok@il\endcsname{\def\PY@tc##1{\textcolor[rgb]{0.40,0.40,0.40}{##1}}}
\expandafter\def\csname PY@tok@mo\endcsname{\def\PY@tc##1{\textcolor[rgb]{0.40,0.40,0.40}{##1}}}
\expandafter\def\csname PY@tok@ch\endcsname{\let\PY@it=\textit\def\PY@tc##1{\textcolor[rgb]{0.25,0.50,0.50}{##1}}}
\expandafter\def\csname PY@tok@cm\endcsname{\let\PY@it=\textit\def\PY@tc##1{\textcolor[rgb]{0.25,0.50,0.50}{##1}}}
\expandafter\def\csname PY@tok@cpf\endcsname{\let\PY@it=\textit\def\PY@tc##1{\textcolor[rgb]{0.25,0.50,0.50}{##1}}}
\expandafter\def\csname PY@tok@c1\endcsname{\let\PY@it=\textit\def\PY@tc##1{\textcolor[rgb]{0.25,0.50,0.50}{##1}}}
\expandafter\def\csname PY@tok@cs\endcsname{\let\PY@it=\textit\def\PY@tc##1{\textcolor[rgb]{0.25,0.50,0.50}{##1}}}

\def\PYZbs{\char`\\}
\def\PYZus{\char`\_}
\def\PYZob{\char`\{}
\def\PYZcb{\char`\}}
\def\PYZca{\char`\^}
\def\PYZam{\char`\&}
\def\PYZlt{\char`\<}
\def\PYZgt{\char`\>}
\def\PYZsh{\char`\#}
\def\PYZpc{\char`\%}
\def\PYZdl{\char`\$}
\def\PYZhy{\char`\-}
\def\PYZsq{\char`\'}
\def\PYZdq{\char`\"}
\def\PYZti{\char`\~}
% for compatibility with earlier versions
\def\PYZat{@}
\def\PYZlb{[}
\def\PYZrb{]}
\makeatother


    % For linebreaks inside Verbatim environment from package fancyvrb.
    \makeatletter
        \newbox\Wrappedcontinuationbox
        \newbox\Wrappedvisiblespacebox
        \newcommand*\Wrappedvisiblespace {\textcolor{red}{\textvisiblespace}}
        \newcommand*\Wrappedcontinuationsymbol {\textcolor{red}{\llap{\tiny$\m@th\hookrightarrow$}}}
        \newcommand*\Wrappedcontinuationindent {3ex }
        \newcommand*\Wrappedafterbreak {\kern\Wrappedcontinuationindent\copy\Wrappedcontinuationbox}
        % Take advantage of the already applied Pygments mark-up to insert
        % potential linebreaks for TeX processing.
        %        {, <, #, %, $, ' and ": go to next line.
        %        _, }, ^, &, >, - and ~: stay at end of broken line.
        % Use of \textquotesingle for straight quote.
        \newcommand*\Wrappedbreaksatspecials {%
            \def\PYGZus{\discretionary{\char`\_}{\Wrappedafterbreak}{\char`\_}}%
            \def\PYGZob{\discretionary{}{\Wrappedafterbreak\char`\{}{\char`\{}}%
            \def\PYGZcb{\discretionary{\char`\}}{\Wrappedafterbreak}{\char`\}}}%
            \def\PYGZca{\discretionary{\char`\^}{\Wrappedafterbreak}{\char`\^}}%
            \def\PYGZam{\discretionary{\char`\&}{\Wrappedafterbreak}{\char`\&}}%
            \def\PYGZlt{\discretionary{}{\Wrappedafterbreak\char`\<}{\char`\<}}%
            \def\PYGZgt{\discretionary{\char`\>}{\Wrappedafterbreak}{\char`\>}}%
            \def\PYGZsh{\discretionary{}{\Wrappedafterbreak\char`\#}{\char`\#}}%
            \def\PYGZpc{\discretionary{}{\Wrappedafterbreak\char`\%}{\char`\%}}%
            \def\PYGZdl{\discretionary{}{\Wrappedafterbreak\char`\$}{\char`\$}}%
            \def\PYGZhy{\discretionary{\char`\-}{\Wrappedafterbreak}{\char`\-}}%
            \def\PYGZsq{\discretionary{}{\Wrappedafterbreak\textquotesingle}{\textquotesingle}}%
            \def\PYGZdq{\discretionary{}{\Wrappedafterbreak\char`\"}{\char`\"}}%
            \def\PYGZti{\discretionary{\char`\~}{\Wrappedafterbreak}{\char`\~}}%
        }
        % Some characters . , ; ? ! / are not pygmentized.
        % This macro makes them "active" and they will insert potential linebreaks
        \newcommand*\Wrappedbreaksatpunct {%
            \lccode`\~`\.\lowercase{\def~}{\discretionary{\hbox{\char`\.}}{\Wrappedafterbreak}{\hbox{\char`\.}}}%
            \lccode`\~`\,\lowercase{\def~}{\discretionary{\hbox{\char`\,}}{\Wrappedafterbreak}{\hbox{\char`\,}}}%
            \lccode`\~`\;\lowercase{\def~}{\discretionary{\hbox{\char`\;}}{\Wrappedafterbreak}{\hbox{\char`\;}}}%
            \lccode`\~`\:\lowercase{\def~}{\discretionary{\hbox{\char`\:}}{\Wrappedafterbreak}{\hbox{\char`\:}}}%
            \lccode`\~`\?\lowercase{\def~}{\discretionary{\hbox{\char`\?}}{\Wrappedafterbreak}{\hbox{\char`\?}}}%
            \lccode`\~`\!\lowercase{\def~}{\discretionary{\hbox{\char`\!}}{\Wrappedafterbreak}{\hbox{\char`\!}}}%
            \lccode`\~`\/\lowercase{\def~}{\discretionary{\hbox{\char`\/}}{\Wrappedafterbreak}{\hbox{\char`\/}}}%
            \catcode`\.\active
            \catcode`\,\active
            \catcode`\;\active
            \catcode`\:\active
            \catcode`\?\active
            \catcode`\!\active
            \catcode`\/\active
            \lccode`\~`\~
        }
    \makeatother

    \let\OriginalVerbatim=\Verbatim
    \makeatletter
    \renewcommand{\Verbatim}[1][1]{%
        %\parskip\z@skip
        \sbox\Wrappedcontinuationbox {\Wrappedcontinuationsymbol}%
        \sbox\Wrappedvisiblespacebox {\FV@SetupFont\Wrappedvisiblespace}%
        \def\FancyVerbFormatLine ##1{\hsize\linewidth
            \vtop{\raggedright\hyphenpenalty\z@\exhyphenpenalty\z@
                \doublehyphendemerits\z@\finalhyphendemerits\z@
                \strut ##1\strut}%
        }%
        % If the linebreak is at a space, the latter will be displayed as visible
        % space at end of first line, and a continuation symbol starts next line.
        % Stretch/shrink are however usually zero for typewriter font.
        \def\FV@Space {%
            \nobreak\hskip\z@ plus\fontdimen3\font minus\fontdimen4\font
            \discretionary{\copy\Wrappedvisiblespacebox}{\Wrappedafterbreak}
            {\kern\fontdimen2\font}%
        }%

        % Allow breaks at special characters using \PYG... macros.
        \Wrappedbreaksatspecials
        % Breaks at punctuation characters . , ; ? ! and / need catcode=\active
        \OriginalVerbatim[#1,codes*=\Wrappedbreaksatpunct]%
    }
    \makeatother

    % Exact colors from NB
    \definecolor{incolor}{HTML}{303F9F}
    \definecolor{outcolor}{HTML}{D84315}
    \definecolor{cellborder}{HTML}{CFCFCF}
    \definecolor{cellbackground}{HTML}{F7F7F7}

    % prompt
    \makeatletter
    \newcommand{\boxspacing}{\kern\kvtcb@left@rule\kern\kvtcb@boxsep}
    \makeatother
    \newcommand{\prompt}[4]{
        \ttfamily\llap{{\color{#2}[#3]:\hspace{3pt}#4}}\vspace{-\baselineskip}
    }



    % Prevent overflowing lines due to hard-to-break entities
    \sloppy
    % Setup hyperref package
    \hypersetup{
      breaklinks=true,  % so long urls are correctly broken across lines
      colorlinks=true,
      urlcolor=urlcolor,
      linkcolor=linkcolor,
      citecolor=citecolor,
      }
    % Slightly bigger margins than the latex defaults

    \geometry{verbose,tmargin=1in,bmargin=1in,lmargin=1in,rmargin=1in}



\begin{document}

    \maketitle




    \hypertarget{introducciuxf3n}{%
\section*{Introducción}\label{introducciuxf3n}}

    En un laboratorio se llevó a cabo un experimento bajo distintas
condiciones para determinar la influencia que el nivel de temperatura y
la naturaleza ácida de cierto caldo de cultivo tienen sobre el
crecimiento de una población de bacterias.

En este documento se plantea, se ajusta y se analiza un modelo de
regresión lineal a partir de los resultados del experimento tomando como
respuesta al número final de bacterias después de 48 horas de una
población inicial de 1000 y como regresores a los factores de
Temperatura (30 ºC, 60 ºC o 90 ºC), tipo de Cultivo (Básico o Ácido) y
Concentración (Baja o Alta) con el propósito de obtener conclusiones
sobre los resultados del experimento realizado.

    \hypertarget{describa-los-datos-gruxe1ficamente.}{%
\section{Describa los datos
gráficamente.}\label{describa-los-datos-gruxe1ficamente.}}

    Como los factores de interés son \texttt{Temperatura} (30 ºC, 60 ºC o 90
ºC) y \texttt{Cultivo} (Básico o Ácido), se considera un total de 6
tratamientos distintos, cada uno con dos niveles de
\texttt{Concentración} (Baja o Alta). En la Figura 1. se encuentran las
gráficas de caja y brazo (\emph{boxplots} en inglés) para cada uno de
los tratamientos considerados. En el eje vertical se encuentra el
\texttt{Número\ final\ de\ bacterias}, es decir, la respuesta, y en el
eje horizontal se encuentran todos los tratamientos tratamientos
considerados junto con una descripción de los parámetros de cada uno.

    \begin{center}
    \adjustimage{max size={0.9\linewidth}{0.9\paperheight}}{Control3_files/Control3_7_0.png}
    \end{center}


    \hypertarget{temperatura}{%
\subparagraph{\texorpdfstring{\texttt{Temperatura}}{Temperatura}}\label{temperatura}}

\begin{itemize}
\item
  \textbf{30 ºC}: En las gráficas de caja y brazo a) y b) de la Figura 1
  se encuentran los resultados para un valor de \texttt{Temperatura} de
  30 ºC. Uniendo los resultados de ambos tipos de \texttt{Cultivo}
  (Ácido y Básico) el \texttt{Número\ final\ de\ bacterias} se encuentra
  entre 93 y 103 unidades y tiene un valor medio de 97.65. En este caso
  la mediana del \texttt{Número\ final\ de\ bacterias} tuvo un
  incremento de 95 a 100 al pasar de un \texttt{Cultivo} Básico a un
  \texttt{Cultivo} Ácido.
\item
  \textbf{60 ºC}: En las gráficas de caja y brazo c) y d) de la Figura 1
  se encuentran los resultados para un valor de \texttt{Temperatura} de
  60 ºC. Uniendo los resultados de ambos tipos de \texttt{Cultivo}
  (Ácido y Básico) el \texttt{Número\ final\ de\ bacterias} se encuentra
  entre 101 y 106 unidades y tiene un valor medio de 102.8. En este caso
  la mediana del \texttt{Número\ final\ de\ bacterias} no tuvo cambio al
  pasar de un \texttt{Cultivo} Básico a un \texttt{Cultivo} Ácido.
\item
  \textbf{90 ºC}: En las gráficas de caja y brazo e) y f) de la Figura 1
  se encuentran los resultados para un valor de \texttt{Temperatura} de
  90 ºC. Uniendo los resultados de ambos tipos de \texttt{Cultivo}
  (Ácido y Básico) el \texttt{Número\ final\ de\ bacterias} se encuentra
  entre 88 y 100 unidades y tiene un valor medio de 94.75. En este caso
  la mediana del \texttt{Número\ final\ de\ bacterias} tuvo un
  incremento de 92.5 a 97 al pasar de un \texttt{Cultivo} Básico a un
  \texttt{Cultivo} Ácido.
\end{itemize}

\hypertarget{comentarios-generales}{%
\subparagraph{Comentarios generales}\label{comentarios-generales}}

El análisis anterior sugiere que el
\texttt{Número\ final\ de\ bacterias} varía dependiendo de la
\texttt{Temperatura} en el sentido de que el
\texttt{Número\ final\ de\ bacterias} tiende a ser mayor si la
\texttt{Temperatura} es de 60 ºC, menor si es de 30 ºC y aún menor si es
de 90 ºC.

Para los niveles de \texttt{Temperatura} de 30 ºC y 90 ºC, el
\texttt{Número\ final\ de\ bacterias} tiende a aumentar si el tipo de
\texttt{Cultivo} cambia de Básico a Ácido. Esto sugiere que el tipo de
\texttt{Cultivo} influye sobre el \texttt{Número\ final\ de\ bacterias}
al menos para estos dos casos. La relación entre \texttt{Cultivo} y
\texttt{Número\ final\ de\ bacterias} no es del todo clara para el caso
de una \texttt{Temperatura} de 60 ºC.

La situación es ambigua con respecto a los niveles de
\texttt{Concentración} ya que, para casi todos los tratamientos, es
posible encontrar observaciones con niveles de \texttt{Concentración}
tanto Baja como Alta en ambos brazos de las gráficas de caja y brazo.
Esto sugiere que el nivel de \texttt{Concentración} no tiene una gran
influencia sobre el \texttt{Número\ final\ de\ bacterias}.

    \hypertarget{puede-considerar-que-hay-diferencia-entre-los-niveles-medios-de-temperatura-esto-es-identifica-efectos-sobre-el-nuxfamero-de-bacterias-por-cambio-en-los-niveles-de-temperatura}{%
\section{¿Puede considerar que hay diferencia entre los niveles medios
de temperatura? Esto es, identifica efectos sobre el número de bacterias
por cambio en los niveles de
temperatura?}\label{puede-considerar-que-hay-diferencia-entre-los-niveles-medios-de-temperatura-esto-es-identifica-efectos-sobre-el-nuxfamero-de-bacterias-por-cambio-en-los-niveles-de-temperatura}}

    Para responder a esta pregunta y a las siguientes cuatro, se propone el
siguiente modelo de regresión lineal.

    \[y = \beta_0 + \beta_1t_1 + \beta_2t_2 + \beta_3k + \beta_4c + \beta_5t_1k + \beta_6t_1c + \beta_7t_2k + \beta_8t_2c + \beta_9kc + \varepsilon\]

donde

\begin{itemize}
\item
  \(y\) es la variable respuesta correspondiente al
  \texttt{Número\ final\ de\ bacterias}.
\item
  \(t_1\) y \(t_2\) son los regresores correspondientes a las variables
  indicadoras asociadas con el nivel de \texttt{Temperatura}, donde \[
  (t_1,\,t_2) =
  \begin{cases}
  (0,\,0) & \text{ si } \texttt{Temperatura} = 30 \text{ ºC} \\
  (1,\,0) & \text{ si } \texttt{Temperatura} = 60 \text{ ºC} \\
  (0,\,1) & \text{ si } \texttt{Temperatura} = 90 \text{ ºC} \\
  \end{cases}
  \]
\item
  \(k\) es el regresor correspondiente a la variable indicadora asociada
  con el tipo de \texttt{Cultivo}, donde \[
  k =
  \begin{cases}
  0 & \text{ si } \texttt{Cultivo} = \text{Básico} \\
  1 & \text{ si } \texttt{Cultivo} = \text{Ácido} \\
  \end{cases}
  \]
\item
  \(c\) es el regresor correspondiente a la variable indicadora asociada
  con el nivel de \texttt{Concentración}, donde \[
  c =
  \begin{cases}
  0 & \text{ si } \texttt{Concentración} = \text{Baja} \\
  1 & \text{ si } \texttt{Concentración} = \text{Alta} \\
  \end{cases}
  \]
\item
  \(\varepsilon\) es el término de error aleatorio.
\end{itemize}

    Es importante resaltar que se pueden obtener 12 modelos diferentes como
resultado de las distintas combinaciones de niveles de
\texttt{Temperatura}, \texttt{Cultivo} y \texttt{Concentración}.

    De modo que

\begin{itemize}
\item
  \(\beta_0\) es el valor medio del
  \texttt{Número\ final\ de\ bacterias} para un nivel de
  \texttt{Temperatura} de 30 ºC con un tipo de \texttt{Cultivo} Básico y
  con \texttt{Concentración} Baja.
\item
  \(\beta_1\) es una medida del cambio en el
  \texttt{Número\ final\ de\ bacterias} al pasar de una
  \texttt{Temperatura} de 30 ºC a una \texttt{Temperatura} de 60 ºC.
\item
  \(\beta_2\) es una medida del cambio en el
  \texttt{Número\ final\ de\ bacterias} al pasar de una
  \texttt{Temperatura} de 30 ºC a una \texttt{Temperatura} de 90 ºC.
  Entonces, la cantidad \(\beta_2\) - \(\beta_1\) es una medida del
  cambio en el \texttt{Número\ final\ de\ bacterias} al pasar de una
  \texttt{Temperatura} de 60 ºC a una \texttt{Temperatura} de 90 ºC.
\item
  \(\beta_3\) es una medida del cambio en el
  \texttt{Número\ final\ de\ bacterias} al pasar de un \texttt{Cultivo}
  Básico a un \texttt{Cultivo} Ácido cuando la \texttt{Temperatura} es
  de 30 ºC.
\item
  \(\beta_4\) es una medida del cambio en el
  \texttt{Número\ final\ de\ bacterias} al pasar de una
  \texttt{Concentración} Baja a una \texttt{Concentración} Alta cuando
  la \texttt{Temperatura} es de 30 ºC.
\item
  \(\beta_5\) mide la interacción entre los regresores \(t_1\) y \(k\).
  La cantidad \(\beta_3 + \beta_5\) es una medida del cambio en el
  \texttt{Número\ final\ de\ bacterias} al pasar de un \texttt{Cultivo}
  Básico a un \texttt{Cultivo} Ácido cuando la \texttt{Temperatura} es
  de 60 ºC.
\item
  \(\beta_6\) mide la interacción entre los regresores \(t_1\) y \(c\).
  La cantidad \(\beta_4 + \beta_6\) es una medida del cambio en el
  \texttt{Número\ final\ de\ bacterias} al pasar de una
  \texttt{Concentración} Baja a una \texttt{Concentración} Alta cuando
  la \texttt{Temperatura} es de 60 ºC.
\item
  \(\beta_7\) mide la interacción entre los regresores \(t_2\) y \(k\).
  La cantidad \(\beta_3 + \beta_7\) es una medida del cambio en el
  \texttt{Número\ final\ de\ bacterias} al pasar de un \texttt{Cultivo}
  Básico a un \texttt{Cultivo} Ácido cuando la \texttt{Temperatura} es
  de 90 ºC.
\item
  \(\beta_8\) mide la interacción entre los regresores \(t_2\) y \(c\).
  La cantidad \(\beta_4 + \beta_8\) es una medida del cambio en el
  \texttt{Número\ final\ de\ bacterias} al pasar de una
  \texttt{Concentración} Baja a una \texttt{Concentración} Alta cuando
  la \texttt{Temperatura} es de 90 ºC.
\item
  \(\beta_9\) mide la interacción entre los regresores \(k\) y \(c\). La
  cantidad \(\beta_9\) es un ajuste en el
  \texttt{Número\ final\ de\ bacterias} cuando el tipo de
  \texttt{Cultivo} es Ácido y la \texttt{Concentración} es Alta.
\end{itemize}

    El ajuste obtenido para el modelo propuesto está dado por

    $$\hat{y} =  94.32 + 8.65{t}_1 - 3.0{t}_2 + 5.37{k} + 1.77{c} - 5.7{t}_1{k} - 1.3{t}_1{c} - 1.6{t}_2{k} + 1.8{t}_2{c} - 0.93{k}{c}$$


    Para el ajuste obtenido se calculó un valor de \(R^2\) de 83.99 \% y un
valor de \(R^2_{\text{ajustada}}\) de 81.11 \%. Esto quiere decir que
los regresores incluidos en el modelo logran explicar una cantidad
suficiente de la variabilidad de la respuesta.


    A continuación se realizan algunas pruebas de hipótesis para análizar
los efectos de los cambios de \texttt{Temperatura} sobre el
\texttt{Número\ final\ de\ bacterias}.

    \begin{itemize}
\tightlist
\item
  Para \(H_0\): \(\beta_0\) = 0 vs.~\(H_1\): \(\beta_0\) \(\neq\) 0 se
  obtuvo un estadístico \(t\) = 124.55 y un valor \(p\) = 1.11
  \(\times\) 10 \(^{-16}\), que es lo suficientemente pequeño como para
  rechazar \(H_0\).
\end{itemize}


    \begin{itemize}
\tightlist
\item
  Para \(H_0\): \(\beta_1\) = 0 vs.~\(H_1\): \(\beta_1\) \(\neq\) 0 se
  obtuvo un estadístico \(t\) = 8.51 y un valor \(p\) = 2.69 \(\times\)
  10 \(^{-11}\), que es lo suficientemente pequeño como para rechazar
  \(H_0\).
\end{itemize}


    \begin{itemize}
\tightlist
\item
  Para \(H_0\): \(\beta_2\) = 0 vs.~\(H_1\): \(\beta_2\) \(\neq\) 0 se
  obtuvo un estadístico \(t\) = -2.95 y un valor \(p\) = 4.78 \(\times\)
  10 \(^{-3}\), que es lo suficientemente pequeño como para rechazar \(H_0\).
\end{itemize}


    De acuerdo con los resultados de las pruebas de hipótesis anteriores, es
posible afirmar que no hay evidencia de que los coeficientes asociados
con los cambios de \texttt{Temperatura} sean negligibles y, por lo
tanto, es posible concluir que los efectos sobre el
\texttt{Número\ final\ de\ bacterias} por cambio en los niveles de
\texttt{Temperatura} sí son identificables.

    \hypertarget{es-distinta-la-respuesta-dependiendo-del-cultivo-empleado}{%
\section{¿Es distinta la respuesta dependiendo del cultivo
empleado?}\label{es-distinta-la-respuesta-dependiendo-del-cultivo-empleado}}

    \begin{itemize}
\tightlist
\item
  Para \(H_0\): \(\beta_3\) = 0 vs.~\(H_1\): \(\beta_3\) \(\neq\) 0 se
  obtuvo un estadístico \(t\) = 5.6 y un valor \(p\) = 9.00 \(\times\)
  10 \(^{-7}\), que es lo suficientemente pequeño como para rechazar \(H_0\).
\end{itemize}


    Como rechazamos la opción que \(\beta_3\) tome el valor de cero, esto
implica que el tipo de \texttt{Cultivo} que se emplea en las muestras sí afecta o
sí tiene un impacto significativo en el \texttt{Número\ final\ de\ bacterias}.

    \hypertarget{consideraruxeda-que-los-factores-interactuxfaan}{%
\section{¿Consideraría que los factores
interactúan?}\label{consideraruxeda-que-los-factores-interactuxfaan}}

    A continuación se harán varias pruebas de hipótesis que ayudarán a
discernir si los distintos factores (\texttt{Temperatura},
\texttt{Cultivo} y \texttt{Concentración}) interactúan entre sí dentro del modelo propuesto para explicar el \texttt{Número\ final\ de\ bacterias}.

    \begin{itemize}
\tightlist
\item
  Para \(H_0\): \(\beta_5\) = 0 vs.~\(H_1\): \(\beta_5\) \(\neq\) 0 se
  obtuvo un estadístico \(t\) = -4.86 y un valor \(p\) = 1.20 \(\times\)
  10 \(^{-5}\), que es lo suficientemente pequeño como para rechazar \(H_0\).
\end{itemize}


    \begin{itemize}
\tightlist
\item
  Para \(H_0\): \(\beta_6\) = 0 vs.~\(H_1\): \(\beta_6\) \(\neq\) 0 se
  obtuvo un estadístico \(t\) = -1.11 y un valor \(p\) = 2.73 \(\times\)
  10 \(^{-1}\), que es lo suficientemente grande como para no rechazar
  \(H_0\).
\end{itemize}


    \begin{itemize}
\tightlist
\item
  Para \(H_0\): \(\beta_7\) = 0 vs.~\(H_1\): \(\beta_7\) \(\neq\) 0 se
  obtuvo un estadístico \(t\) = -1.36 y un valor \(p\) = 1.78 \(\times\)
  10 \(^{-1}\), que es lo suficientemente grande como para no rechazar
  \(H_0\).
\end{itemize}


    \begin{itemize}
\tightlist
\item
  Para \(H_0\): \(\beta_8\) = 0 vs.~\(H_1\): \(\beta_8\) \(\neq\) 0 se
  obtuvo un estadístico \(t\) = 1.53 y un valor \(p\) = 1.31 \(\times\)
  10 \(^{-1}\), que es lo suficientemente grande como para no rechazar
  \(H_0\).
\end{itemize}


    \begin{itemize}
\tightlist
\item
  Para \(H_0\): \(\beta_9\) = 0 vs.~\(H_1\): \(\beta_9\) \(\neq\) 0 se
  obtuvo un estadístico \(t\) = -0.97 y un valor \(p\) = 3.34 \(\times\)
  10 \(^{-1}\), que es lo suficientemente grande como para no rechazar
  \(H_0\).
\end{itemize}


    Como podemos observar, todos los valores $p$ de las pruebas de hipótesis
son valores grandes (mayores a 0.05), con excepción del valor de la
prueba para \(\beta_5\). Sin embargo, como \(\beta_5\) indica la
interacción entre el cambio de 30ºC a 60ºC de \texttt{Temperatura} y el
tipo de \texttt{Cultivo}, también tenemos que tomar en cuenta
\(\beta_6\) para modelar la interacción total entre estas dos variables.

    \hypertarget{valide-su-modelo-mediante-el-anuxe1lisis-de-los-residuales.}{%
\section{Valide su modelo mediante el análisis de los
residuales.}\label{valide-su-modelo-mediante-el-anuxe1lisis-de-los-residuales.}}

    Realizaremos un análisis de residuales para ver si el modelo cumple con
todos los supuestos: 1. Varianza constante 2. Errores independientes 3.
Errores se distribuyen normal con media 0

    \begin{center}
    \adjustimage{max size={0.9\linewidth}{0.9\paperheight}}{Control3_files/Control3_37_0.png}
    \end{center}


    En la Figura 2 vemos que el modelo sí cumple con el supuesto de
homocedasticidad pues la varianza de los residuales es prácticamente
constante. Al inicio tiene una menor varianza sin embargo después se
estabiliza y se mantene constante.

    \begin{center}
    \adjustimage{max size={0.9\linewidth}{0.9\paperheight}}{Control3_files/Control3_39_0.png}
    \end{center}


    En la Figura 3 vemos que, aunque no es idéntica, la distribución de los
residuales se asemeja bastante a la de una normal. Esto se confirma con
la gráfica cuantil-cuantil de la Figura 4 ya que, salvo al inicio y al
final, los residuales sí se ajustan a la recta y por lo tanto cumplen
con el supuesto de normalidad.

    \begin{center}
    \adjustimage{max size={0.9\linewidth}{0.9\paperheight}}{Control3_files/Control3_41_0.png}
    \end{center}


    Finalmente con la Figura 5 y la Figura 6 vemos que los residuales no
presentan ninguna correlación aparente entre ellos y que la correlación
de orden uno es muy pequeña por lo que también se cumple el supuesto de
independencia.

Una vez que nuestros residuales sí cumplen los 3 supuestos podemos
afirmar que nuestro modelo es válido y útil para analizar nuestros
datos.

    \begin{center}
    \adjustimage{max size={0.9\linewidth}{0.9\paperheight}}{Control3_files/Control3_44_0.png}
    \end{center}


    Finalmente de la Figura 7 vemos que los residuales estandarizados son
menores que 1.75 y mayores que 0 por lo que podemos concluir que nuestra
muestra no cuenta con valores atípicos.

    \hypertarget{considera-que-es-necesario-distinguir-entre-los-niveles-bajo-y-alto-de-la-concentraciuxf3n-del-cultivo}{%
\section{¿Considera que es necesario distinguir entre los niveles bajo y
alto de la concentración del
cultivo?}\label{considera-que-es-necesario-distinguir-entre-los-niveles-bajo-y-alto-de-la-concentraciuxf3n-del-cultivo}}

    Para \(H_0: \beta_4 = 0\)
vs.~\(H_1: \beta_4 \neq 0\) se obtuvo un estadístico \(t_4 = 1.84\) y un valor \(p = 7.11 \times 10^{-2}\), que es lo suficientemente grande como para no rechazar la hipótesis \(H_0\). Esto significa que el
coeficiente \(\beta_4\) no es significativo y por lo tanto no es
necesario distinguir entre los niveles de concentración ya que no nos
brinda información extra.

    \hypertarget{concluya-el-reporte-resaltando-lo-encontrado-en-su-anuxe1lisis.}{%
\section{Concluya el reporte resaltando lo encontrado en su
análisis.}\label{concluya-el-reporte-resaltando-lo-encontrado-en-su-anuxe1lisis.}}


%    \begin{verbatim}
%            df           F      PR(>F)
%t1         1.0  1.6880e+02  1.1941e-17
%t2         1.0  2.4443e+01  9.0172e-06
%k          1.0  2.6526e+01  4.4140e-06
%t1:k       1.0  2.3261e+01  1.3658e-05
%t2:k       1.0  1.8601e+00  1.7872e-01
%t2:c       1.0  1.3965e+01  4.7895e-04
%t1:c       1.0  5.7349e-28  1.0000e+00
%c          1.0  2.4559e+00  1.2339e-01
%k:c        1.0  9.4943e-01  3.3456e-01
%Residual  50.0         NaN         NaN
%    \end{verbatim
\begin{center}
    Tabla \ref{seq}. Análisis de varianza por medio de suma extra de cuadrados.
\end{center}

\begin{table}[!htbp]
\centering
\caption{Análisis de varianza.}
\label{seq}
\begin{tabular}{cccccc}
\toprule
Variación &  Suma de cuadrados &  Grados de libertad &  Cuadrados medios &  $F_0$ &                  Valor $p$ \\
\midrule
$t_1$      &             580.80 &                   1 &            580.80 & 168.80 &  $1.11 \times 10 ^{-16}$ \\
$t_2$       &              84.10 &                   1 &             84.10 &  24.44 &   $9.01 \times 10 ^{-6}$ \\
$k$        &              91.27 &                   1 &             91.27 &  26.53 &   $4.41 \times 10^{-6}$ \\
$t_1k$  &              80.03 &                   1 &             80.03 &  23.26 &   $1.36 \times 10^{-5}$ \\
$t_2k$  &               6.40 &                   1 &              6.40 &   1.86 &   $1.78 \times 10^{-1}$ \\
$t_2c$  &              48.05 &                   1 &             48.05 &  13.97 &   $4.78 \times 10^{-4}$ \\
$t_1c$  &               0.00 &                   1 &              0.00 &   0.00 &   $9.99 \times 10^{-1}$ \\
$c$         &               8.45 &                   1 &              8.45 &   2.46 &   $1.23 \times 10^{-1}$ \\
$kc$    &               3.27 &                   1 &              3.27 &   0.95 &   $3.34 \times 10^{-1}$ \\
Residuales   &             172.03 &                  50 &              3.44 &        &                            \\
\bottomrule
\end{tabular}
\end{table}

    Realizamos un análisis de varianza por medio de las pruebas de suma extra de cuadrados (Ver Tabla \ref{seq}) para ver cuáles regresores podemos
eliminar de nuestro modelo. Como se observó en las pruebas de hipótesis
realizadas en secciones anteriores tanto la \texttt{Temperatura} como
el tipo de \texttt{Cultivo}, así como su interacción, son variables
importantes para determinar el \texttt{Número\ final\ de\ bacterias}.

Adicionalmente, con este análisis observamos que el regresor $t_2c$ asociado con la interacción entre la \texttt{Temperatura} y la \texttt{Concentración} también es un regresor
importante para explicar nuestros datos ya que en la Tabla \ref{seq} tiene un valor $p$ asociado de
\(4.78 \times 10 ^{-4} < 0.05\). Es importante notar que la prueba de
hipótesis de la pregunta 4 nos indica que si el regresor $t_2c$ es el
último regresor en agregarse al modelo, entonces, dados los regresores anteriores, no es muy
significativo. Sin embargo, la tabla anterior nos indica que si en nuestro modelo solo
tomamos en cuenta la \texttt{Temperatura}, el \texttt{Cultivo} y las respectivas interacciones entre estos regresores, agregar el regresor $t_2c$ sí nos ayuda a
explicar de mejor manera nuestros datos. Por el \emph{Principio de
Herencia} al agregar la interacción entre \texttt{Temperatura} y
\texttt{Concentración} también debemos de agregar el regresor de
\texttt{Concentración} a la muestra aunque éste no sea tan
significativo.

    Por lo tanto, el modelo seleccionado para explicar el
\texttt{Número\ final\ de\ bacterias} es de la forma:
\[y=\beta_0 + \beta_1 t_1 + \beta_2 t_2 + \beta_3 k + \beta_4 c + \beta_5 t_1k + \beta_6 t_2k + \beta_7 t_1c + \beta_8 t_2c + \varepsilon\]
y el modelo ajustado es:
\[\hat{y}= 94.55 + 8.65t_1 - 3.00t_2 + 4.90k + 1.30c - 5.70t_1k -1.60t_2k - 1.30t_1c + 1.80t_2c \]

    \begin{center}
    \adjustimage{max size={0.9\linewidth}{0.9\paperheight}}{Control3_files/Control3_54_0.png}
    \end{center}


    En la Figura 8 se encuentran graficados los intervalos de 95 \% de confianza del valor promedio de la respuesta para cada uno de los tratamientos considerados. Vemos que el ajuste asociado con nuestro modelo final es satisfactorio ya que las medias observadas sí están dentro de los
intervalos de confianza de las predicciones del modelo.

Notamos que las bacterias reaccionan mejor a una temperatura de 60ºC sin
importar el tipo de cultivo ya que entre el cultivo ácido y básico hay
menos de un punto de diferencia. Además en el caso de los 60ºC no
importa la concentración inicial del experimento pues con ambas se
obtienen los mismos resultados. Por el otro lado, para temperaturas de
30ºC y 90ºC la bacterias crecen mejor con un cultivo ácido y
concentraciones iniciales altas, pero prefieren temperaturas más bajas
(30ºC). Todo esto concuerda con nuestras observaciones iniciales.


    % Add a bibliography block to the postdoc



\end{document}
