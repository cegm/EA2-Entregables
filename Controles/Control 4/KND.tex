\documentclass[11pt]{article}

    \usepackage[breakable]{tcolorbox}
    \usepackage{parskip} % Stop auto-indenting (to mimic markdown behaviour)
    \usepackage{iftex}
    \ifPDFTeX
    	\usepackage[T1]{fontenc}
    	\usepackage{mathpazo}
    \else
    	\usepackage{fontspec}
    \fi

    % Basic figure setup, for now with no caption control since it's done
    % automatically by Pandoc (which extracts ![](path) syntax from Markdown).
    \usepackage{graphicx}
    % Maintain compatibility with old templates. Remove in nbconvert 6.0
    \let\Oldincludegraphics\includegraphics
    % Ensure that by default, figures have no caption (until we provide a
    % proper Figure object with a Caption API and a way to capture that
    % in the conversion process - todo).
    \usepackage{caption}
    %\DeclareCaptionFormat{nocaption}{}
    %captionsetup{format=nocaption,aboveskip=0pt,belowskip=0pt}

    \usepackage[Export]{adjustbox} % Used to constrain images to a maximum size
    \adjustboxset{max size={0.9\linewidth}{0.9\paperheight}}
    \usepackage{float}
    \floatplacement{figure}{H} % forces figures to be placed at the correct location
    \usepackage{xcolor} % Allow colors to be defined
    \usepackage{enumerate} % Needed for markdown enumerations to work
    \usepackage{geometry} % Used to adjust the document margins
    \usepackage{amsmath} % Equations
    \usepackage{amssymb} % Equations
    \usepackage{textcomp} % defines textquotesingle
    % Hack from http://tex.stackexchange.com/a/47451/13684:
    \AtBeginDocument{%
        \def\PYZsq{\textquotesingle}% Upright quotes in Pygmentized code
    }
    \usepackage{upquote} % Upright quotes for verbatim code
    \usepackage{eurosym} % defines \euro
    \usepackage[mathletters]{ucs} % Extended unicode (utf-8) support
    \usepackage{fancyvrb} % verbatim replacement that allows latex
    \usepackage{grffile} % extends the file name processing of package graphics 
                         % to support a larger range
    \makeatletter % fix for grffile with XeLaTeX
    \def\Gread@@xetex#1{%
      \IfFileExists{"\Gin@base".bb}%
      {\Gread@eps{\Gin@base.bb}}%
      {\Gread@@xetex@aux#1}%
    }
    \makeatother

    % The hyperref package gives us a pdf with properly built
    % internal navigation ('pdf bookmarks' for the table of contents,
    % internal cross-reference links, web links for URLs, etc.)
    \usepackage{hyperref}
    % The default LaTeX title has an obnoxious amount of whitespace. By default,
    % titling removes some of it. It also provides customization options.
    \usepackage{titling}
    \usepackage{longtable} % longtable support required by pandoc >1.10
    \usepackage{booktabs}  % table support for pandoc > 1.12.2
    \usepackage[inline]{enumitem} % IRkernel/repr support (it uses the enumerate* environment)
    \usepackage[normalem]{ulem} % ulem is needed to support strikethroughs (\sout)
                                % normalem makes italics be italics, not underlines
    \usepackage{mathrsfs}
    

    
    % Colors for the hyperref package
    \definecolor{urlcolor}{rgb}{0,.145,.698}
    \definecolor{linkcolor}{rgb}{.71,0.21,0.01}
    \definecolor{citecolor}{rgb}{.12,.54,.11}

    % ANSI colors
    \definecolor{ansi-black}{HTML}{3E424D}
    \definecolor{ansi-black-intense}{HTML}{282C36}
    \definecolor{ansi-red}{HTML}{E75C58}
    \definecolor{ansi-red-intense}{HTML}{B22B31}
    \definecolor{ansi-green}{HTML}{00A250}
    \definecolor{ansi-green-intense}{HTML}{007427}
    \definecolor{ansi-yellow}{HTML}{DDB62B}
    \definecolor{ansi-yellow-intense}{HTML}{B27D12}
    \definecolor{ansi-blue}{HTML}{208FFB}
    \definecolor{ansi-blue-intense}{HTML}{0065CA}
    \definecolor{ansi-magenta}{HTML}{D160C4}
    \definecolor{ansi-magenta-intense}{HTML}{A03196}
    \definecolor{ansi-cyan}{HTML}{60C6C8}
    \definecolor{ansi-cyan-intense}{HTML}{258F8F}
    \definecolor{ansi-white}{HTML}{C5C1B4}
    \definecolor{ansi-white-intense}{HTML}{A1A6B2}
    \definecolor{ansi-default-inverse-fg}{HTML}{FFFFFF}
    \definecolor{ansi-default-inverse-bg}{HTML}{000000}

    % commands and environments needed by pandoc snippets
    % extracted from the output of `pandoc -s`
    \providecommand{\tightlist}{%
      \setlength{\itemsep}{0pt}\setlength{\parskip}{0pt}}
    \DefineVerbatimEnvironment{Highlighting}{Verbatim}{commandchars=\\\{\}}
    % Add ',fontsize=\small' for more characters per line
    \newenvironment{Shaded}{}{}
    \newcommand{\KeywordTok}[1]{\textcolor[rgb]{0.00,0.44,0.13}{\textbf{{#1}}}}
    \newcommand{\DataTypeTok}[1]{\textcolor[rgb]{0.56,0.13,0.00}{{#1}}}
    \newcommand{\DecValTok}[1]{\textcolor[rgb]{0.25,0.63,0.44}{{#1}}}
    \newcommand{\BaseNTok}[1]{\textcolor[rgb]{0.25,0.63,0.44}{{#1}}}
    \newcommand{\FloatTok}[1]{\textcolor[rgb]{0.25,0.63,0.44}{{#1}}}
    \newcommand{\CharTok}[1]{\textcolor[rgb]{0.25,0.44,0.63}{{#1}}}
    \newcommand{\StringTok}[1]{\textcolor[rgb]{0.25,0.44,0.63}{{#1}}}
    \newcommand{\CommentTok}[1]{\textcolor[rgb]{0.38,0.63,0.69}{\textit{{#1}}}}
    \newcommand{\OtherTok}[1]{\textcolor[rgb]{0.00,0.44,0.13}{{#1}}}
    \newcommand{\AlertTok}[1]{\textcolor[rgb]{1.00,0.00,0.00}{\textbf{{#1}}}}
    \newcommand{\FunctionTok}[1]{\textcolor[rgb]{0.02,0.16,0.49}{{#1}}}
    \newcommand{\RegionMarkerTok}[1]{{#1}}
    \newcommand{\ErrorTok}[1]{\textcolor[rgb]{1.00,0.00,0.00}{\textbf{{#1}}}}
    \newcommand{\NormalTok}[1]{{#1}}
    
    % Additional commands for more recent versions of Pandoc
    \newcommand{\ConstantTok}[1]{\textcolor[rgb]{0.53,0.00,0.00}{{#1}}}
    \newcommand{\SpecialCharTok}[1]{\textcolor[rgb]{0.25,0.44,0.63}{{#1}}}
    \newcommand{\VerbatimStringTok}[1]{\textcolor[rgb]{0.25,0.44,0.63}{{#1}}}
    \newcommand{\SpecialStringTok}[1]{\textcolor[rgb]{0.73,0.40,0.53}{{#1}}}
    \newcommand{\ImportTok}[1]{{#1}}
    \newcommand{\DocumentationTok}[1]{\textcolor[rgb]{0.73,0.13,0.13}{\textit{{#1}}}}
    \newcommand{\AnnotationTok}[1]{\textcolor[rgb]{0.38,0.63,0.69}{\textbf{\textit{{#1}}}}}
    \newcommand{\CommentVarTok}[1]{\textcolor[rgb]{0.38,0.63,0.69}{\textbf{\textit{{#1}}}}}
    \newcommand{\VariableTok}[1]{\textcolor[rgb]{0.10,0.09,0.49}{{#1}}}
    \newcommand{\ControlFlowTok}[1]{\textcolor[rgb]{0.00,0.44,0.13}{\textbf{{#1}}}}
    \newcommand{\OperatorTok}[1]{\textcolor[rgb]{0.40,0.40,0.40}{{#1}}}
    \newcommand{\BuiltInTok}[1]{{#1}}
    \newcommand{\ExtensionTok}[1]{{#1}}
    \newcommand{\PreprocessorTok}[1]{\textcolor[rgb]{0.74,0.48,0.00}{{#1}}}
    \newcommand{\AttributeTok}[1]{\textcolor[rgb]{0.49,0.56,0.16}{{#1}}}
    \newcommand{\InformationTok}[1]{\textcolor[rgb]{0.38,0.63,0.69}{\textbf{\textit{{#1}}}}}
    \newcommand{\WarningTok}[1]{\textcolor[rgb]{0.38,0.63,0.69}{\textbf{\textit{{#1}}}}}
    
    
    % Define a nice break command that doesn't care if a line doesn't already
    % exist.
    \def\br{\hspace*{\fill} \\* }
    % Math Jax compatibility definitions
    \def\gt{>}
    \def\lt{<}
    \let\Oldtex\TeX
    \let\Oldlatex\LaTeX
    \renewcommand{\TeX}{\textrm{\Oldtex}}
    \renewcommand{\LaTeX}{\textrm{\Oldlatex}}
    % Document parameters
    % Document title
    \title{Control 4}
    \author{KND}
    \date{25 de noviembre de 2020}
    
    
    
    
    
% Pygments definitions
\makeatletter
\def\PY@reset{\let\PY@it=\relax \let\PY@bf=\relax%
    \let\PY@ul=\relax \let\PY@tc=\relax%
    \let\PY@bc=\relax \let\PY@ff=\relax}
\def\PY@tok#1{\csname PY@tok@#1\endcsname}
\def\PY@toks#1+{\ifx\relax#1\empty\else%
    \PY@tok{#1}\expandafter\PY@toks\fi}
\def\PY@do#1{\PY@bc{\PY@tc{\PY@ul{%
    \PY@it{\PY@bf{\PY@ff{#1}}}}}}}
\def\PY#1#2{\PY@reset\PY@toks#1+\relax+\PY@do{#2}}

\expandafter\def\csname PY@tok@w\endcsname{\def\PY@tc##1{\textcolor[rgb]{0.73,0.73,0.73}{##1}}}
\expandafter\def\csname PY@tok@c\endcsname{\let\PY@it=\textit\def\PY@tc##1{\textcolor[rgb]{0.25,0.50,0.50}{##1}}}
\expandafter\def\csname PY@tok@cp\endcsname{\def\PY@tc##1{\textcolor[rgb]{0.74,0.48,0.00}{##1}}}
\expandafter\def\csname PY@tok@k\endcsname{\let\PY@bf=\textbf\def\PY@tc##1{\textcolor[rgb]{0.00,0.50,0.00}{##1}}}
\expandafter\def\csname PY@tok@kp\endcsname{\def\PY@tc##1{\textcolor[rgb]{0.00,0.50,0.00}{##1}}}
\expandafter\def\csname PY@tok@kt\endcsname{\def\PY@tc##1{\textcolor[rgb]{0.69,0.00,0.25}{##1}}}
\expandafter\def\csname PY@tok@o\endcsname{\def\PY@tc##1{\textcolor[rgb]{0.40,0.40,0.40}{##1}}}
\expandafter\def\csname PY@tok@ow\endcsname{\let\PY@bf=\textbf\def\PY@tc##1{\textcolor[rgb]{0.67,0.13,1.00}{##1}}}
\expandafter\def\csname PY@tok@nb\endcsname{\def\PY@tc##1{\textcolor[rgb]{0.00,0.50,0.00}{##1}}}
\expandafter\def\csname PY@tok@nf\endcsname{\def\PY@tc##1{\textcolor[rgb]{0.00,0.00,1.00}{##1}}}
\expandafter\def\csname PY@tok@nc\endcsname{\let\PY@bf=\textbf\def\PY@tc##1{\textcolor[rgb]{0.00,0.00,1.00}{##1}}}
\expandafter\def\csname PY@tok@nn\endcsname{\let\PY@bf=\textbf\def\PY@tc##1{\textcolor[rgb]{0.00,0.00,1.00}{##1}}}
\expandafter\def\csname PY@tok@ne\endcsname{\let\PY@bf=\textbf\def\PY@tc##1{\textcolor[rgb]{0.82,0.25,0.23}{##1}}}
\expandafter\def\csname PY@tok@nv\endcsname{\def\PY@tc##1{\textcolor[rgb]{0.10,0.09,0.49}{##1}}}
\expandafter\def\csname PY@tok@no\endcsname{\def\PY@tc##1{\textcolor[rgb]{0.53,0.00,0.00}{##1}}}
\expandafter\def\csname PY@tok@nl\endcsname{\def\PY@tc##1{\textcolor[rgb]{0.63,0.63,0.00}{##1}}}
\expandafter\def\csname PY@tok@ni\endcsname{\let\PY@bf=\textbf\def\PY@tc##1{\textcolor[rgb]{0.60,0.60,0.60}{##1}}}
\expandafter\def\csname PY@tok@na\endcsname{\def\PY@tc##1{\textcolor[rgb]{0.49,0.56,0.16}{##1}}}
\expandafter\def\csname PY@tok@nt\endcsname{\let\PY@bf=\textbf\def\PY@tc##1{\textcolor[rgb]{0.00,0.50,0.00}{##1}}}
\expandafter\def\csname PY@tok@nd\endcsname{\def\PY@tc##1{\textcolor[rgb]{0.67,0.13,1.00}{##1}}}
\expandafter\def\csname PY@tok@s\endcsname{\def\PY@tc##1{\textcolor[rgb]{0.73,0.13,0.13}{##1}}}
\expandafter\def\csname PY@tok@sd\endcsname{\let\PY@it=\textit\def\PY@tc##1{\textcolor[rgb]{0.73,0.13,0.13}{##1}}}
\expandafter\def\csname PY@tok@si\endcsname{\let\PY@bf=\textbf\def\PY@tc##1{\textcolor[rgb]{0.73,0.40,0.53}{##1}}}
\expandafter\def\csname PY@tok@se\endcsname{\let\PY@bf=\textbf\def\PY@tc##1{\textcolor[rgb]{0.73,0.40,0.13}{##1}}}
\expandafter\def\csname PY@tok@sr\endcsname{\def\PY@tc##1{\textcolor[rgb]{0.73,0.40,0.53}{##1}}}
\expandafter\def\csname PY@tok@ss\endcsname{\def\PY@tc##1{\textcolor[rgb]{0.10,0.09,0.49}{##1}}}
\expandafter\def\csname PY@tok@sx\endcsname{\def\PY@tc##1{\textcolor[rgb]{0.00,0.50,0.00}{##1}}}
\expandafter\def\csname PY@tok@m\endcsname{\def\PY@tc##1{\textcolor[rgb]{0.40,0.40,0.40}{##1}}}
\expandafter\def\csname PY@tok@gh\endcsname{\let\PY@bf=\textbf\def\PY@tc##1{\textcolor[rgb]{0.00,0.00,0.50}{##1}}}
\expandafter\def\csname PY@tok@gu\endcsname{\let\PY@bf=\textbf\def\PY@tc##1{\textcolor[rgb]{0.50,0.00,0.50}{##1}}}
\expandafter\def\csname PY@tok@gd\endcsname{\def\PY@tc##1{\textcolor[rgb]{0.63,0.00,0.00}{##1}}}
\expandafter\def\csname PY@tok@gi\endcsname{\def\PY@tc##1{\textcolor[rgb]{0.00,0.63,0.00}{##1}}}
\expandafter\def\csname PY@tok@gr\endcsname{\def\PY@tc##1{\textcolor[rgb]{1.00,0.00,0.00}{##1}}}
\expandafter\def\csname PY@tok@ge\endcsname{\let\PY@it=\textit}
\expandafter\def\csname PY@tok@gs\endcsname{\let\PY@bf=\textbf}
\expandafter\def\csname PY@tok@gp\endcsname{\let\PY@bf=\textbf\def\PY@tc##1{\textcolor[rgb]{0.00,0.00,0.50}{##1}}}
\expandafter\def\csname PY@tok@go\endcsname{\def\PY@tc##1{\textcolor[rgb]{0.53,0.53,0.53}{##1}}}
\expandafter\def\csname PY@tok@gt\endcsname{\def\PY@tc##1{\textcolor[rgb]{0.00,0.27,0.87}{##1}}}
\expandafter\def\csname PY@tok@err\endcsname{\def\PY@bc##1{\setlength{\fboxsep}{0pt}\fcolorbox[rgb]{1.00,0.00,0.00}{1,1,1}{\strut ##1}}}
\expandafter\def\csname PY@tok@kc\endcsname{\let\PY@bf=\textbf\def\PY@tc##1{\textcolor[rgb]{0.00,0.50,0.00}{##1}}}
\expandafter\def\csname PY@tok@kd\endcsname{\let\PY@bf=\textbf\def\PY@tc##1{\textcolor[rgb]{0.00,0.50,0.00}{##1}}}
\expandafter\def\csname PY@tok@kn\endcsname{\let\PY@bf=\textbf\def\PY@tc##1{\textcolor[rgb]{0.00,0.50,0.00}{##1}}}
\expandafter\def\csname PY@tok@kr\endcsname{\let\PY@bf=\textbf\def\PY@tc##1{\textcolor[rgb]{0.00,0.50,0.00}{##1}}}
\expandafter\def\csname PY@tok@bp\endcsname{\def\PY@tc##1{\textcolor[rgb]{0.00,0.50,0.00}{##1}}}
\expandafter\def\csname PY@tok@fm\endcsname{\def\PY@tc##1{\textcolor[rgb]{0.00,0.00,1.00}{##1}}}
\expandafter\def\csname PY@tok@vc\endcsname{\def\PY@tc##1{\textcolor[rgb]{0.10,0.09,0.49}{##1}}}
\expandafter\def\csname PY@tok@vg\endcsname{\def\PY@tc##1{\textcolor[rgb]{0.10,0.09,0.49}{##1}}}
\expandafter\def\csname PY@tok@vi\endcsname{\def\PY@tc##1{\textcolor[rgb]{0.10,0.09,0.49}{##1}}}
\expandafter\def\csname PY@tok@vm\endcsname{\def\PY@tc##1{\textcolor[rgb]{0.10,0.09,0.49}{##1}}}
\expandafter\def\csname PY@tok@sa\endcsname{\def\PY@tc##1{\textcolor[rgb]{0.73,0.13,0.13}{##1}}}
\expandafter\def\csname PY@tok@sb\endcsname{\def\PY@tc##1{\textcolor[rgb]{0.73,0.13,0.13}{##1}}}
\expandafter\def\csname PY@tok@sc\endcsname{\def\PY@tc##1{\textcolor[rgb]{0.73,0.13,0.13}{##1}}}
\expandafter\def\csname PY@tok@dl\endcsname{\def\PY@tc##1{\textcolor[rgb]{0.73,0.13,0.13}{##1}}}
\expandafter\def\csname PY@tok@s2\endcsname{\def\PY@tc##1{\textcolor[rgb]{0.73,0.13,0.13}{##1}}}
\expandafter\def\csname PY@tok@sh\endcsname{\def\PY@tc##1{\textcolor[rgb]{0.73,0.13,0.13}{##1}}}
\expandafter\def\csname PY@tok@s1\endcsname{\def\PY@tc##1{\textcolor[rgb]{0.73,0.13,0.13}{##1}}}
\expandafter\def\csname PY@tok@mb\endcsname{\def\PY@tc##1{\textcolor[rgb]{0.40,0.40,0.40}{##1}}}
\expandafter\def\csname PY@tok@mf\endcsname{\def\PY@tc##1{\textcolor[rgb]{0.40,0.40,0.40}{##1}}}
\expandafter\def\csname PY@tok@mh\endcsname{\def\PY@tc##1{\textcolor[rgb]{0.40,0.40,0.40}{##1}}}
\expandafter\def\csname PY@tok@mi\endcsname{\def\PY@tc##1{\textcolor[rgb]{0.40,0.40,0.40}{##1}}}
\expandafter\def\csname PY@tok@il\endcsname{\def\PY@tc##1{\textcolor[rgb]{0.40,0.40,0.40}{##1}}}
\expandafter\def\csname PY@tok@mo\endcsname{\def\PY@tc##1{\textcolor[rgb]{0.40,0.40,0.40}{##1}}}
\expandafter\def\csname PY@tok@ch\endcsname{\let\PY@it=\textit\def\PY@tc##1{\textcolor[rgb]{0.25,0.50,0.50}{##1}}}
\expandafter\def\csname PY@tok@cm\endcsname{\let\PY@it=\textit\def\PY@tc##1{\textcolor[rgb]{0.25,0.50,0.50}{##1}}}
\expandafter\def\csname PY@tok@cpf\endcsname{\let\PY@it=\textit\def\PY@tc##1{\textcolor[rgb]{0.25,0.50,0.50}{##1}}}
\expandafter\def\csname PY@tok@c1\endcsname{\let\PY@it=\textit\def\PY@tc##1{\textcolor[rgb]{0.25,0.50,0.50}{##1}}}
\expandafter\def\csname PY@tok@cs\endcsname{\let\PY@it=\textit\def\PY@tc##1{\textcolor[rgb]{0.25,0.50,0.50}{##1}}}

\def\PYZbs{\char`\\}
\def\PYZus{\char`\_}
\def\PYZob{\char`\{}
\def\PYZcb{\char`\}}
\def\PYZca{\char`\^}
\def\PYZam{\char`\&}
\def\PYZlt{\char`\<}
\def\PYZgt{\char`\>}
\def\PYZsh{\char`\#}
\def\PYZpc{\char`\%}
\def\PYZdl{\char`\$}
\def\PYZhy{\char`\-}
\def\PYZsq{\char`\'}
\def\PYZdq{\char`\"}
\def\PYZti{\char`\~}
% for compatibility with earlier versions
\def\PYZat{@}
\def\PYZlb{[}
\def\PYZrb{]}
\makeatother


    % For linebreaks inside Verbatim environment from package fancyvrb. 
    \makeatletter
        \newbox\Wrappedcontinuationbox 
        \newbox\Wrappedvisiblespacebox 
        \newcommand*\Wrappedvisiblespace {\textcolor{red}{\textvisiblespace}} 
        \newcommand*\Wrappedcontinuationsymbol {\textcolor{red}{\llap{\tiny$\m@th\hookrightarrow$}}} 
        \newcommand*\Wrappedcontinuationindent {3ex } 
        \newcommand*\Wrappedafterbreak {\kern\Wrappedcontinuationindent\copy\Wrappedcontinuationbox} 
        % Take advantage of the already applied Pygments mark-up to insert 
        % potential linebreaks for TeX processing. 
        %        {, <, #, %, $, ' and ": go to next line. 
        %        _, }, ^, &, >, - and ~: stay at end of broken line. 
        % Use of \textquotesingle for straight quote. 
        \newcommand*\Wrappedbreaksatspecials {% 
            \def\PYGZus{\discretionary{\char`\_}{\Wrappedafterbreak}{\char`\_}}% 
            \def\PYGZob{\discretionary{}{\Wrappedafterbreak\char`\{}{\char`\{}}% 
            \def\PYGZcb{\discretionary{\char`\}}{\Wrappedafterbreak}{\char`\}}}% 
            \def\PYGZca{\discretionary{\char`\^}{\Wrappedafterbreak}{\char`\^}}% 
            \def\PYGZam{\discretionary{\char`\&}{\Wrappedafterbreak}{\char`\&}}% 
            \def\PYGZlt{\discretionary{}{\Wrappedafterbreak\char`\<}{\char`\<}}% 
            \def\PYGZgt{\discretionary{\char`\>}{\Wrappedafterbreak}{\char`\>}}% 
            \def\PYGZsh{\discretionary{}{\Wrappedafterbreak\char`\#}{\char`\#}}% 
            \def\PYGZpc{\discretionary{}{\Wrappedafterbreak\char`\%}{\char`\%}}% 
            \def\PYGZdl{\discretionary{}{\Wrappedafterbreak\char`\$}{\char`\$}}% 
            \def\PYGZhy{\discretionary{\char`\-}{\Wrappedafterbreak}{\char`\-}}% 
            \def\PYGZsq{\discretionary{}{\Wrappedafterbreak\textquotesingle}{\textquotesingle}}% 
            \def\PYGZdq{\discretionary{}{\Wrappedafterbreak\char`\"}{\char`\"}}% 
            \def\PYGZti{\discretionary{\char`\~}{\Wrappedafterbreak}{\char`\~}}% 
        } 
        % Some characters . , ; ? ! / are not pygmentized. 
        % This macro makes them "active" and they will insert potential linebreaks 
        \newcommand*\Wrappedbreaksatpunct {% 
            \lccode`\~`\.\lowercase{\def~}{\discretionary{\hbox{\char`\.}}{\Wrappedafterbreak}{\hbox{\char`\.}}}% 
            \lccode`\~`\,\lowercase{\def~}{\discretionary{\hbox{\char`\,}}{\Wrappedafterbreak}{\hbox{\char`\,}}}% 
            \lccode`\~`\;\lowercase{\def~}{\discretionary{\hbox{\char`\;}}{\Wrappedafterbreak}{\hbox{\char`\;}}}% 
            \lccode`\~`\:\lowercase{\def~}{\discretionary{\hbox{\char`\:}}{\Wrappedafterbreak}{\hbox{\char`\:}}}% 
            \lccode`\~`\?\lowercase{\def~}{\discretionary{\hbox{\char`\?}}{\Wrappedafterbreak}{\hbox{\char`\?}}}% 
            \lccode`\~`\!\lowercase{\def~}{\discretionary{\hbox{\char`\!}}{\Wrappedafterbreak}{\hbox{\char`\!}}}% 
            \lccode`\~`\/\lowercase{\def~}{\discretionary{\hbox{\char`\/}}{\Wrappedafterbreak}{\hbox{\char`\/}}}% 
            \catcode`\.\active
            \catcode`\,\active 
            \catcode`\;\active
            \catcode`\:\active
            \catcode`\?\active
            \catcode`\!\active
            \catcode`\/\active 
            \lccode`\~`\~ 	
        }
    \makeatother

    \let\OriginalVerbatim=\Verbatim
    \makeatletter
    \renewcommand{\Verbatim}[1][1]{%
        %\parskip\z@skip
        \sbox\Wrappedcontinuationbox {\Wrappedcontinuationsymbol}%
        \sbox\Wrappedvisiblespacebox {\FV@SetupFont\Wrappedvisiblespace}%
        \def\FancyVerbFormatLine ##1{\hsize\linewidth
            \vtop{\raggedright\hyphenpenalty\z@\exhyphenpenalty\z@
                \doublehyphendemerits\z@\finalhyphendemerits\z@
                \strut ##1\strut}%
        }%
        % If the linebreak is at a space, the latter will be displayed as visible
        % space at end of first line, and a continuation symbol starts next line.
        % Stretch/shrink are however usually zero for typewriter font.
        \def\FV@Space {%
            \nobreak\hskip\z@ plus\fontdimen3\font minus\fontdimen4\font
            \discretionary{\copy\Wrappedvisiblespacebox}{\Wrappedafterbreak}
            {\kern\fontdimen2\font}%
        }%
        
        % Allow breaks at special characters using \PYG... macros.
        \Wrappedbreaksatspecials
        % Breaks at punctuation characters . , ; ? ! and / need catcode=\active 	
        \OriginalVerbatim[#1,codes*=\Wrappedbreaksatpunct]%
    }
    \makeatother

    % Exact colors from NB
    \definecolor{incolor}{HTML}{303F9F}
    \definecolor{outcolor}{HTML}{D84315}
    \definecolor{cellborder}{HTML}{CFCFCF}
    \definecolor{cellbackground}{HTML}{F7F7F7}
    
    % prompt
    \makeatletter
    \newcommand{\boxspacing}{\kern\kvtcb@left@rule\kern\kvtcb@boxsep}
    \makeatother
    \newcommand{\prompt}[4]{
        \ttfamily\llap{{\color{#2}[#3]:\hspace{3pt}#4}}\vspace{-\baselineskip}
    }
    

    
    % Prevent overflowing lines due to hard-to-break entities
    \sloppy 
    % Setup hyperref package
    \hypersetup{
      breaklinks=true,  % so long urls are correctly broken across lines
      colorlinks=true,
      urlcolor=urlcolor,
      linkcolor=linkcolor,
      citecolor=citecolor,
      }
    % Slightly bigger margins than the latex defaults
    
    \geometry{verbose,tmargin=1in,bmargin=1in,lmargin=1in,rmargin=1in}
    
    

\begin{document}
    
    \maketitle
    
    

    
    \hypertarget{introducciuxf3n}{%
\section{Introducción}\label{introducciuxf3n}}

    Los datos analizados en el artículo {[}1{]} se componen de distintas
variables relacionadas con el avance tecnológico de varios vehículos
híbridos eléctricos. En este documento se utiliza el mismo conjunto de
datos para construir, ajustar y validar un modelo de regresión lineal
múltiple que explique el precio de los vehículos en términos de
variables como la tasa de aceleración y el consumo de combustible entre
otras. El ajuste obtenido es utilizado posteriormente para la predicción
del precio de los vehículos de otro conjunto de datos.

    \hypertarget{resumen-del-artuxedculo}{%
\section{Resumen del artículo}\label{resumen-del-artuxedculo}}

    En el artículo {[}1{]} se realiza un pronóstico de tecnología por medio
de análisis por envolvente de datos (TFDEA por sus siglas en inglés)
cuyo objetivo es medir y comparar el avance tecnológico ocurrido entre
2004 y 2013 en distintos sectores del mercado de vehículos híbridos
eléctricos.

Los autores del artículo consideran como variable de entrada el precio
sugerido por el fabricante de los vehículos. Como variables de salida
consideran la tasa de aceleración, el consumo de combustible y el máximo
de consumo de combustible o equivalente. También se incluye una variable
categórica que divide a los vehículos en siete grupos diferentes.

De acuerdo con los resultados obtenidos en el artículo, los sectores
correspondientes a los vehículos de tamaño medio fueron los que tuvieron
mayores avances tecnológicos durante el periodo estudiado. Las mejoras
en el desempeño y la diversificación de estos últimos puede representar
una amenaza para los vehículos de tamaño pequeño. Mientras que los
vehículos utilitarios deportivos (SUV) se enfocan en un nicho de mercado
lujoso, los vehículos especializados para el transporte de mercancías
compiten contra sus equivalentes de gasolina para probar la utilidad de
los vehículos híbridos.

    \hypertarget{descripciuxf3n-de-las-variables-del-conjunto-de-datos-y-anuxe1lisis-exploratorio}{%
\section{Descripción de las variables del conjunto de datos y análisis
exploratorio}\label{descripciuxf3n-de-las-variables-del-conjunto-de-datos-y-anuxe1lisis-exploratorio}}

    El conjunto de datos cuenta con las siguientes variables

\textbf{Variable respuesta}

La variable de respuesta que se busca es el precio sugerido del vehículo
(\texttt{msrp} por sus siglas en inglés), que toma valores decimales y
que en los datos originales está expresada en dólares equivalentes a
2013. Los valores mínimo y máximo que se tienen son 11849.43 y 118543.6
respectivamente. Para expresar esta variable en miles de dólares, lo que
hacemos es dividir esa columna entre 1000.

\textbf{Potenciales regresores}

\begin{itemize}
\item
  tasa de aceleración (\texttt{accelrate}): Esta variable numérica
  indica el tiempo en segundos que le toma a cada vehículo llegar de 0
  km/h a 100 km/h.
\item
  consumo de combustible (\texttt{mpg}): También conocida como economía
  de combustible, esta variable numéroca indica la distancia que un
  vehículo puede recorrer por unidad de combustible. En la fuente
  original de los datos se encuentra en millas por galón. En este
  documento se convirtió a kilómetros por litro multiplicando por
  2.8248.
\item
  máximo de consumo de combustible o equivalente (\texttt{mpgmpge}): La
  Agencia de Protección Ambiental de Estados Unidos desarrolló un
  consumo de combustible equivalente para los vehículos híbridos
  eléctricos. Esta variable indica el máximo entre el consumo de
  combustible tradicional y el equivalente. También se convirtió a
  kilómetros por litro.
\item
  modelo (\texttt{vehicle}): Es una variable categórica que indica el
  modelo del y que toma 109 valores diferentes.
\item
  clase del vehículo (\texttt{carclass}): Es la categoría a la cual
  pertenece cada vehículo (detalle más adelante).
\end{itemize}

A pesar de las transformaciones realizadas se decidió mantener el nombre
original de las variables. Además los datos cuentan con otras variables:
\texttt{carid}, \texttt{year} y \texttt{carclass\_id} que son variables
que no aportan mucho al análisis de los datos ya que son para el manejo
interno de los datos o fungen como etiquetas.

    La base de datos se compone de 140 observaciones de diferentes coches y
no tenemos datos faltantes ni repetidos. Existen 109 modelos de coches
distintos y 7 clases.

    \begin{center}
    \adjustimage{max size={0.9\linewidth}{0.9\paperheight}}{KND_files/KND_12_0.png}
    \end{center}
    
    
    De la Figura 1 vemos que hay más carros de tamaño medio, seguido por las
SUV y los autos compactos, por lo que la mayoría de los datos de la base
son coches familiares donde caben más de 3 personas. Creamos una nueva
variable para las clases de coche que agrupa los coches similares en una
clase. Los nuevos grupos están dados por:

\begin{itemize}
\item
  Carros grandes: Large y Pickup Truck (\texttt{Large})
\item
  Camionetas: SUV y Minivan (\texttt{Vans})
\item
  Carros pequeños: Compact y Two-seaters (\texttt{Small})
\item
  Carros medianos: Midsize (\texttt{Medium})
\end{itemize}

En lo subsecuente, la variable \texttt{carclass} se refiere a esta nueva
asignación de categorías.

    De igual manera se creó una nueva variable para el modelo de coche que
agrupa los modelos por marca, por ejemplo: \emph{Prius},
\emph{Highlander} y \emph{Camry} son de la marca \emph{Toyota}. De esta
manera, la nueva variable \texttt{manufacturer} se redujo a 25 clases en
lugar de las 103 de la variable \texttt{vehicle}. En la Figura 2 vemos
que \emph{Toyota}, \emph{Honda}, \emph{Lexus}, \emph{Chevrolet} y
\emph{Ford} son las marcas con más observaciones.

    \begin{center}
    \adjustimage{max size={0.9\linewidth}{0.9\paperheight}}{KND_files/KND_18_0.png}
    \end{center}
    
    
    Para la mayoría de las observaciones el consumo de combustible es igual
al máximo entre el \texttt{mpg} y la medición alternativa del consumo de
combustible. Sin embargo, hay 11 observaciones en las que la medición
alternativa es más del doble de \texttt{mpg} (Figura 3).

    A continuación evaluaremos la relación lineal entre los regresores y la
variable respuesta.

    \begin{center}
    \adjustimage{max size={0.9\linewidth}{0.9\paperheight}}{KND_files/KND_21_0.png}
    \end{center}
    
    
    \begin{center}
    \adjustimage{max size={0.9\linewidth}{0.9\paperheight}}{KND_files/KND_22_0.png}
    \end{center}
    
    
    La tasa de aceleración es la única que presenta una correlación lineal
clara con el precio (Figura 4). Tanto para el consumo de combustible
como para el máximo de consumo de combustible o equivalente, la relación
con la variable objetivo es no-lineal (Figuras 5 y 6). También podemos
ver que las variables \texttt{mpg} y \texttt{mpgmpge} tienen un
comportamiento similar, lo cual tiene sentido ya que solo difieren en 11
observaciones.

Esto se puede confirmar en la Figura 7 ya que \texttt{accelrate} tiene
una correlación de \(0.7\) con la variable objetivo, mientras que para
\texttt{mpg} y \texttt{mpgmpge}, además de que las correlaciones son
negativas, estas no son muy grandes en magnitud.

    \hypertarget{modelo-de-regresiuxf3n-lineal-muxfaltiple}{%
\section{Modelo de regresión lineal
múltiple}\label{modelo-de-regresiuxf3n-lineal-muxfaltiple}}

    \hypertarget{modelo-inicial}{%
\subsection{Modelo inicial}\label{modelo-inicial}}

    En primer lugar, se ajustó un modelo de regresión lineal tomando como
respuesta a la variable \texttt{msrp} y como regresores a las variables
numéricas \texttt{accelrate}, \texttt{mpg} y \texttt{mpgmpge} y a los
factores \texttt{carclass}, \texttt{manufacturer}.

    Vemos que \texttt{mpg} no es significativo pues tiene un valor-p
\(0.081 > 0.05\)

    
    \begin{center}
    \adjustimage{max size={0.9\linewidth}{0.9\paperheight}}{KND_files/KND_29_0.png}
    \end{center}
    
    
    En la Figura 8 se encuentran graficados los residuales contra las
respuestas ajustadas del obtenidas a partir del modelo inicial. La
gráfica sugiere una desviación notable del supuesto de varianza
constante de los términos aleatorios de error.

    \hypertarget{transformaciuxf3n-de-variables}{%
\subsection{Transformación de
variables}\label{transformaciuxf3n-de-variables}}

    Con el objetivo de mitigar las desviaciones de los supuestos del modelo
de regresión lineal se utilizó la transformación de Box-Cox en la
variable respuesta. Para obtener un valor de \(\lambda\) óptimo se
consideraron varios valores de prueba. Con cada uno de ellos se obtuvo
un ajuste y se seleccionó el valor de \(\lambda\) que minimizara la suma
de cuadrados de los residuales de cada uno de los ajustes. En la Tabla 1
se muestra que el valor óptimo resultó ser \(\lambda = 0\), lo cual
sugiere que la varianza de los términos de error podría estabilizarse al
aplicar una transformación logarítmica a la variable respuesta.

\begin{table}[!h]
    \centering
    \begin{tabular}{rrl}
        \toprule
        $\lambda$ &  $SC_{Res}(\lambda)$ & Clasificación \\
        \midrule
        -1.00 &             10269.7716 &               \\
        -0.75 &              8874.2238 &               \\
        -0.50 &              7989.9042 &               \\
        -0.25 &              7546.4395 &               \\
        \textbf{0.00} &     \textbf{7525.9739} &        \textbf{Óptimo}\\
        0.25 &              7963.9670 &               \\
        0.50 &              8959.1730 &               \\
        0.75 &             10695.3245 &               \\
        1.00 &             13479.5726 &               \\
        \bottomrule
    \end{tabular}
    \caption*{Tabla 1. Valores de prueba de $\lambda$ para la transformación de Box-Cox}
\end{table}
    
    
    Para la selección del modelo final se utilizó un procedimiento
computacional que consiste en evaluar todas las posibles regresiones.
Los parámetros de evaluación fueron el Criterio de Información de Akaike
(AIC) y la \(C_p\) de Mallows y el valor \(p\) asociado con el
estadístico \(F\) de la significancia de la regresión.

\begin{table}[!h]
    \centering
    \begin{tabular}{lcccc}
        \toprule
        model &  num\_regresors &      AIC &  Mallows Cp &   f\_pvalue \\
        \midrule
        mpg &              2 &  156.077 &         2.0 &  $2.31\times10^{-11}$ \\
        mpgmpge &              2 &  185.046 &         2.0 &  $5.61\times10^{-5}$ \\
        carclass &              4 &  164.835 &         4.0 &  $1.23\times10^{-8}$\\
        accelrate &              2 &  109.445 &         2.0 &  $1.85\times10^{-21}$ \\
        manufacturer &             25 &   59.143 &        25.0 &  $1.59\times10^{-23}$\\
        mpg+carclass &              5 &  152.226 &         5.0 &  $5.94\times10^{-11}$ \\
        mpg+accelrate &              3 &  100.821 &         3.0 &  $1.47\times10^{-22}$\\
        mpg+manufacturer &             26 &   53.568 &        26.0 & $3.36\times10^{-24}$ \\
        carclass+accelrate &              5 &   94.008 &         5.0 &  $6.40\times10^{-23}$\\
        carclass+manufacturer &             28 &   50.344 &        28.0 &  $3.20\times10^{-24}$\\
        accelrate+manufacturer &             26 &   32.315 &        26.0 &  $9.74\times10^{-28}$\\
        mpg+mpgmpge+manufacturer &             27 &   48.724 &        27.0 &  $9.66\times10^{-25}$\\
        mpgmpge+carclass+manufacturer &             29 &   46.563 &        29.0 &  $1.41\times10^{-24}$\\
        carclass+accelrate+manufacturer &             29 &   28.718 &        29.0 &  $1.84\times10^{-27}$\\
        mpg+mpgmpge+carclass+manufacturer &             30 &   38.505 &        30.0 &  $1.33\times10^{-25}$\\
        mpg+mpgmpge+accelrate+manufacturer &             28 &   29.926 &        28.0 &  $1.49\times10^{-27}$\\
        mpgmpge+carclass+accelrate+manufacturer &             30 &   \textbf{21.086} &        30.0 &  $2.11\times10^{-28}$\\
        \bottomrule
        \end{tabular}
        \caption*{Tabla 2. Estadísticos relevantes para la selección del modelo final}
\end{table}

    
    En la Tabla 2 se muestra que el modelo que simultáneamente minimiza el
valor del AIC y el valor \(p\) asociado con el estadístico \(F\) de la
prueba de la significancia de la regresión es el que se muestra en la
última fila de la tabla, es decir, aquel que considera a
\texttt{mpgmpge}, \texttt{carclass}, \texttt{accelrate} y
\texttt{manufacturer}como regresores. En todos los casos la \(C_p\) de
Mallows es igual a la cantidad de regresores empleados. La cantidad de
30 regresores corresponde a (25 - 1) variables indicadoras asociadas con
el factor \texttt{manufacturer}, (4 - 1) variables indicadoras asociadas
con el factor \texttt{carclass}, 2 variables numéricas y 1 regresor
asociado con el coeficiente \(\beta_0\)

    \hypertarget{detecciuxf3n-de-valores-atuxedpicos}{%
\subsection{Detección de valores
atípicos}\label{detecciuxf3n-de-valores-atuxedpicos}}

    Para la detección de valores atípicos se calcularon los residuales
estandarizados (\(\hat{r}_i\)) y las distancias de Cook (\(d_i\)). Con
base en el criterio \(|\hat{r}_i| > 3\) se detectaron dos valores
atípicos: El vehículo cuyo modelo es \texttt{Crown} y el vehículo cuyo
modelo es \texttt{Lexus\ LS600h/hL}. Esta situación se encuentra
ilustrada en la Figura 9. No se detectaron valore de influencia con el
criterio \(d_i > 1\).

    \begin{center}
    \adjustimage{max size={0.9\linewidth}{0.9\paperheight}}{KND_files/KND_45_0.png}
    \end{center}
    
    
    \begin{table}[!hb]
        \centering
        \begin{tabular}{rrrlrll}
            \toprule
            msrp &      mpg &  mpgmpge & carclass &  accelrate & manufacturer &          vehicle \\
            \midrule
            2.4723 &  14.9991 &  14.9991 &   Medium &       7.87 &         Audi &           A5 BSG \\
            2.6801 &  13.2985 &  13.2985 &   Medium &       7.14 &      Besturn &    Besturn B50   \\
            3.0134 &  19.5566 &  19.5566 &   Medium &       9.90 &       Toyota &            Prius \\
            3.1409 &  12.8181 &  36.1372 &   Medium &       9.24 &          BYD &        F3DM PHEV \\
            3.1673 &  12.8011 &  36.1372 &   Medium &       9.52 &          BYD &             F3DM \\
            3.1864 &  21.2572 &  21.2572 &   Medium &      10.20 &       Toyota &            Prius \\
            3.2096 &  12.3292 &  12.3292 &   Medium &       9.09 &    Chevrolet &           Malibu \\
            3.2790 &  15.3052 &  15.3052 &   Medium &      10.54 &          Kia &        Optima K5 \\
            3.3059 &  14.0000 &  14.0000 &   Medium &       9.51 &       Toyota &          Prius V \\
            3.4206 &  31.0015 &  31.0015 &   Medium &      10.00 &       Toyota &  Prius alpha (V) \\
            3.4657 &  21.2572 &  40.3887 &   Medium &       9.17 &       Toyota &    Prius Plug-in \\
            3.4687 &  21.2572 &  40.3887 &   Medium &       8.82 &       Toyota &        Prius PHV \\
            \textbf{4.1318} &  \textbf{15.7984} &  \textbf{15.7984} &   \textbf{Medium} &       \textbf{8.70} &       \textbf{Toyota} &            \textbf{Crown} \\
            \bottomrule
        \end{tabular}
        \caption*{Tabla 3. Vehículos con \texttt{accelrate} entre 6.70 y 10.70 y con \texttt{carcalss} == "Medium"}
    \end{table}

    
    En la Tabla 3 se muestra que el vehículo \texttt{Crown} es el único de
la clase \texttt{Medium} cuyo \texttt{accelrate} se encuentra entre 6.70
y 10.70 que tiene un precio en miles de dólares de 2013 \texttt{msrp}
mayor a 40.

Además de que una situación similar se presenta para el caso del
vehículo \texttt{Lexus\ LS600h/hL}, sucede que este es el vehículo más
caro de todo el conjunto de datos con un \texttt{msrp} de 118.5436 miles
de dólares de 2013.

Se ha decidido conservar estas observaciones debido a que no hay
evidencia suficiente para concluir que la atipicidad de sus valores
pueda ser atribuida a errores incurridos durante la recolección de los
datos.

    \hypertarget{validaciuxf3n-del-modelo}{%
\subsubsection{Validación del modelo}\label{validaciuxf3n-del-modelo}}

    \begin{center}
    \adjustimage{max size={0.9\linewidth}{0.9\paperheight}}{KND_files/KND_49_0.png}
    \end{center}
    
    
    En la Figura 10 se encuentran graficados los residuales contra las
respuestas ajustadas obtenidas a partir del modelo final seleccionado.
En comparación con la Figura 8, el problema de heteroscedasticidad fue
ligeramente mitigado.

    \begin{center}
    \adjustimage{max size={0.9\linewidth}{0.9\paperheight}}{KND_files/KND_51_0.png}
    \end{center}
    
    
    En la Figura 11 se encuentra un histograma de los resiudales que sugiere
que, salvo por algunas observaciones, el supuesto de normalidad no está
siendo del todo violado. Esto se evidencia con mayor claridad en la
Figura 12, que es una gráfica cuantil-cuantil de los residuales contra
la escala normal. Salvo por los valores influyentes mencionados
previamente, la desviación del supuesto de normalidad no es muy grave.

    \begin{center}
    \adjustimage{max size={0.9\linewidth}{0.9\paperheight}}{KND_files/KND_53_0.png}
    \end{center}
    
    
    Finalmente con la Figura 13 y la Figura 14 vemos que los residuales no
presentan ninguna correlación aparente entre ellos y que la correlación
de orden uno es muy pequeña por lo que también se cumple el supuesto de
independencia.

    \hypertarget{predicciuxf3n-del-precio-de-nuevos-vehuxedculos}{%
\section{Predicción del precio de nuevos
vehículos}\label{predicciuxf3n-del-precio-de-nuevos-vehuxedculos}}

    Recordemos que nuestro modelo está dado por:

    \begin{equation}
    y = 2.5233 + \begin{pmatrix}
       -0.2331 \\
       -0.2592 \\ 
       0.0038
    \end{pmatrix}' \begin{pmatrix}
       Medium \\
       Small \\ 
       Vans
    \end{pmatrix} + \begin{pmatrix}
        0.7806 \\
       -0.0849 \\ 
       -0.1878 \\
        0.2666 \\
        1.0763 \\
        0.3199 \\
        0.3493 \\
        0.0759 \\
        0.5968 \\
       -0.0630 \\
        0.0789 \\
        0.3755 \\
       -0.6023 \\
        0.1295 \\
        0.5186 \\
        0.3632 \\
       -0.1374 \\
        0.8580 \\
        0.2471 \\
       -0.1051 \\
        0.5763 \\
        0.6483 \\
        0.2047 \\
        0.3373
    \end{pmatrix}' \begin{pmatrix}
       BMW \\
       BYD \\
       Besturn \\
       Buick \\
       Cadillac \\
       Chevrolet \\
       Dodge \\
       Ford \\
       GMC \\
       Honda \\
       Hyundai \\
       Infiniti \\
       Jeep \\
       Kia \\
       Lexus \\
       Lincoln \\
       Mazda \\
       Mercedes-Benz \\
       Nissan \\
       Opel \\
       Peugeot \\
       Porsche \\
       Toyota \\
       Volkswagen
    \end{pmatrix} + 0.0102*mpgmpge + 0.0620*accelrate
\end{equation}

    Para poder predecir los nuevos datos que tenemos, primero se tienen que
hacer las mismas modificaciones que se hicieron con los datos de
entrenamiento. Es decir, dividir el \texttt{msrp} entre 1000, expresar
las columnas \texttt{mpg} y \texttt{mpgmpge} en términos de kilómetros
por litro y clasificar las clases y las marcas de los vehículos como se
hizo previamente. Una vez realizados estos cambios, procedemos a hacer
la predicción y a analizar los resultados obtenidos.

    \begin{center}
    \adjustimage{max size={0.9\linewidth}{0.9\paperheight}}{KND_files/KND_65_0.png}
    \end{center}
    
    
    En la Figura 15 se pueden observar los resultados obtenidos con el
modelo. En la mayoría de los casos se obtuvo una buena predicción ya que
los datos observados están dentro del intervalo de confianza de las
predicciones del modelo. Finamlente se calculó la \(R^2_{prediccion}\)
mediante la fórmula \(R^2_{prediccion} = 1 - \frac{PRESS}{SS_T}\) y se
obtuvo un valor de 0.9078. Lo que afirma que es una buena predicción de
los datos nuevos.

    \hypertarget{referencias}{%
\section{Referencias}\label{referencias}}

    {[}1{]} Lim, D., Jahromi, S., Anderson, T. and Tudorie, A., 2015.
\emph{Comparing technological advancement of hybrid electric vehicles
(HEV) in different market segments}. Technological Forecasting and
Social Change, 97, pp.140-153.


    % Add a bibliography block to the postdoc
    
    
    
\end{document}
